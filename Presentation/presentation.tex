\documentclass{beamer}
\graphicspath{{../graphics/}}
\usepackage{listings}
\usepackage{ulem}

\newcommand{\linespace}{\vspace{1em}}

\mode<presentation>
{
  \usetheme{Darmstadt}
  \setbeamertemplate{footline}[frame number]
  \setbeamertemplate{navigation symbols}{}
  \setbeamercovered{transparent}
}

\AtBeginSection[]
{
   \begin{frame}
        \frametitle{Table of Contents}
        \tableofcontents[sectionstyle=show/hide,subsectionstyle=show/show/hide]
   \end{frame}
}

%\usepackage[danish]{babel}
\usepackage[T1]{fontenc}

\usepackage[utf8]{inputenc}

\usepackage{times}

\usepackage{tikz}

\title[Mapping med Lego-robot]{Mapping med Lego-robot}

\subtitle{SW505E13}

\author[SW505E13]{Mikkel Sand\o ~Larsen, \and Bruno Thalmann, \and Stefan Marstrand Getreuer Micheelsen, \and Stefan Thilemann, \and Mikael Elki\ae r Christensen, \and Anders R. Nielsen}

\institute[Aalborg University]
{
  Department of Computer Science\\
  Aalborg University}

\date[CFP 2003]{31. Januar 2014}

\begin{document}

%--------------------------------------------------
%     INTRODUKTION
%--------------------------------------------------

\begin{frame}
  \titlepage
\end{frame}

\begin{frame}
    \frametitle{Table of Contents}
    \tableofcontents[sectionstyle=show/show,subsectionstyle=hide/hide/hide]
\end{frame}

\section{Overblik}

\subsection{Intro test}
\begin{frame}
\includegraphics{systemarkitektur_1.png}
\end{frame}
\end{document}