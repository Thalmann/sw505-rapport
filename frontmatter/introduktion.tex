Robotter bliver i større og større omfang benyttet til at automatisere komplicerede opgaver.
Således kan man bruge robotter til lagerstyring, til at udforske et ufremkommeligt område eller f.eks. også til at hjælpe ældre i deres hverdag.

En del af disse problemer kræver at robotten har et kort over det område den skal arbejde i.
En lagerrobot kræver et kort over lageret så den ikke kører ind i vægge når den fragter objekter rundt.
En støvsugerrobot skal kende de rum den skal støvsuge så den kan navigere rundt om stoleben, og stadig få støvsuget hele rummet.

For at lave dette kort kan robotten udstyres med sensorer og ud fra de opsamlede data konstruere et kort.

Dette problem fører til den initierende problemstilling:
\quoter{Er det muligt at kortlægge et ukendt areal ved hjælp af en robot?}

Denne rapport vil undersøge hvordan dette problem løses og beskrive den relevante teori.
Baseret på denne teori vil der blive udviklet et system der kortlægger et rum ved at bruge sensormålinger opsamlet af en robot der kører rundt i rummet.
