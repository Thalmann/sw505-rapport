\section{Lokalisering}
Når robotten skal kortlægge sine omgivelser er det nødvendigt at vide hvor robotten befinder sig.
Til at finde robottens position har vi overvejet flere forskellige metoder.
\begin{itemize}
\item Kinect
\item Dead reckoning
\item Kamera ansigtsgenkendelse
\end{itemize}

\subsection{Kinect }
Microsofts Kinect benytter et rgb kamera samt en dybdesensor til at fortolke bevægelse som input i applikationer.
Kinect er oprindeligt lavet til XBox 360, men findes også i en version til PC, med medfølgende SDK.
En stor fordel ved at bruge Kinect er at den indeholder en dybdesensor der kan bruges til at lokalisere objekter i et rum.

\subsection{Wiimote}
En anden sensor der kan bruges er Nintendos Wiimote. 
Wiimote en optisk sensor til at lokalisere infrarødt lys fra den tilhørende sensor bar.
Ved hjælp af triangulering af det modtagne lys kan wiimoten bestemme sin position.

\subsection{Dead reckoning}
Dead Reckoning er en teknik til at finde sin position efter ens bevægelser uden eksterne sensorer.
Robotten ved hvor den starter, og beregner sin nye position efter hvilken kommando den har udført.\cite{deadrec}

Denne teknik kan risikere at være meget upræcis, hvis robottens motorer og sensorer mangler præcision.

\subsection{Valg af lokaliseringsmetode}
Der blev foretaget en indledende test af de tre foreslåede lokaliseringsmetoder for at vælge hvilken der passede bedst til dette projekt.

Dead reckoning muligheden blev hurtigt udelukket, da gruppen ville fokusere på kortlægningsopgaven og dead reckoning ville kræve en del mere arbejde end de andre lokaliseringsmetoder.

Valget mellem Kinect og Wiimote vandt Kinect af flere grunde. 
For det første havde vi adgang til en Kinect hvilket gjorde valget oplagt. 
Også 