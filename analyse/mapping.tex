\section{Introduktion til mapping}
Behovet for at fremskaffe et kort over en robots omgivelser vil i den ene eller anden forstand altid være påkrævet før robotten er i stand til at interagere med det miljø den er placeret i.
Det kan f.eks. være et stort udendørs areal man ønsker at bygge et kort over, eller f.eks. robottens eget syn på den verden hvori den bevæger sig.

Der findes flere afprøvede metoder indenfor robotik der gør det muligt for en robot at navigere og bygge kort over dens umiddelbare nærmeste omgivelser.
Dette afsnit vil fokusere på to af disse, navnligt \textit{Occupancy Grid} og \textit{Particle Filters}, og give en overordnet beskrivelse der skal danne grundlag for valget af metode for den fortløbende udvikling af projektet.
Afslutningsvis vil de to metoder blive sammenlignet for at belyse eventuelle fordele og ulemper ved begge.

\subsection{Occupancy Grid}


\subsection{Particle Filters}

\subsection{Sammenligning}