
\chapter{LEGO MINDSTORMS NXT}
I dette kapitel vil den valgte platform, \legoms, blive kort beskrevet, samt en begrundelse for dette valg.
Yderligere vil der blive argumenteret for valg af API til NXT-enheden.

\section{Hvad er LEGO MINDSTORMS?}
\legoms er et byggesæt, hvor det er muligt at bygge programmerbare robotter(her menes også andre maskiner der måske ikke altid vil blive betegnet som robotter) i \lego-klodser.

Til at bygge disse robotter er der i MINDSTORMS\texttrademark nogle sensorer og aktuatorer. Sensorerne gør det muligt for robotten at modtage input fra sine omgivelser og ved brug af aktuatorerne kan robotten reagere på disse.

Ud over de originale \lego dele er der også tredjeparts forhandlere, som har et udbud af andre sensorer og aktuatorer.

\subsection{Hvad er NXT?}
Denne sektion er baseret på \cite{nxt}.
NXT Intelligent Brick (oftest kaldt blot 'NXT' eller 'brick') er hjernen i \legoms robotten.
Det er den der står for at modtage og behandle input fra sensorer, samt at styre aktuatorer.

\paragraph{Porte}
NXT'en har 3 motor porte og 4 sensor porte.

\paragraph{Tilslutningsmuligheder}
Der kan kommunikeres med NXT'en enten ved at tilslutte den med USB-kabel eller ved Bluetooth\textregistered.

\paragraph{Feedback}
Til output har NXT'en en 100 x 64 pixel LCD display samt 8 kHz højttaler.

\paragraph{Styring}
NXT'en kan styres på to måder:
Man kan sende kommandoer og modtage beskeder (for eksempel sensor aflæsninger) på en ekstern enhed (oftest en computer eller en anden NXT).
Alternativt kan programmer sendes (via Bluetooth eller USB) til NXT'en, hvorfra de kan køres direkte på NXT'en, uafhængig af eksterne enheder.

Til at styre NXT'en kan bruges et væld af programmeringssprog.
Yderligere er det også muligt at erstatte den originale firmware med en brugerdefineret, hvilket giver endnu flere muligheder.
Kombination af firmware og programmeringssprog (samt eventuelle værktøjer) er hvad vi betragter som et API.
De forskellige API'er samt vores valg af API diskuteres i \cref{nxt_api}.

\section{Hvorfor LEGO MINDSTORMS NXT?}

\chapter{Valg af NXT API}
\label{nxt_api}

\section{De forskellige API'er}

\section{Vores valg}
