\section{Lokalisering af robot}
\mikkel{Hvor gør vi opmærksom på at vi vil anvende en kinect?}
Til lokalisering af robotten anvendes (som nævnt i \cref{}) en \kinect.
Denne er udstyret med et farvekamera (jvf. \cref{kinect:farvekamera}) der vil blive anvendt til at bestemme robottens placering.
Ved at udstyre robotten med to mærkninger i klare farver vil det på billeder fra kinect'en være muligt at lokalisere robotten.
I det følgende beskrives den anvendte metode til lokalisering af robotten.
Først beskrives den overordnede metode, hvorefter forbedringer af metoden præsenteres.

\subsection{Farveforskel}
%Beskrivelse af den metode der anvendes til at beskrive forskellen mellem farver

\subsection{Farve-afstand}
%Beskrivelse af farve-afstand optimering, ved opdatering af den søgte farve

\subsection{Afgrænset område}
%Beskrivelse af opdatering af det område der afsøges

\subsection{Justering}
%Beskrivelse af diverse thresholds og hvilke værdier "der fungerer"
\mikkel{Resultater af colortracking-test}