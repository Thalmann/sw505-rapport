\section{HiTechnic NXT Compass Sensor}
Kompasset fra HiTechnic er et digitalt kompas der måles jordens magnetiske felt, hvor kompasset kan returnere en værdi der repræsenterer den nuværende retning (heading).
Kompasset er præcist ned til 1 grad (repræsenteret ved en værdi mellem 0-179) og opdaterer sin heading op til 100 gange i sekundet.

Kalibrering er ikke nødvendigt.
Dog er det nødvendigt for at minimere forstyrrelser, at kompasset holdes 10-15 cm væk fra forstyrrende elementer, herunder \lego NXT og motorer.\cite{hitechnic_compass}

\subsection{Compass Sensor i \mindsqualls}
Til at styre denne sensor i \mindsqualls, skal der bruges en af de seperate klasser til HiTechnic sensorerne, som findes i namespace \lstinline[style=csharp]!NXT.MindSqualls.HiTechnic!.
Til Compass Sensor hedder klassen \lstinline[style=csharp]!HiTechnicCompassSensor!.

\subsubsection{Aflæsning af værdier}
Compass Sensor aflæses som alle andre sensorer, da den også arver fra \lstinline[style=csharp]!Pollable!-klassen.
Ved aflæsning af egenskaben \lstinline[style=csharp]!Heading!, gives et lige heltal mellem 0 og 358 (180 værdier i alt).
Et eksempel på en simpel aflæsning af værdi kan ses i \cref{kompas:kode_eksempel}.

\begin{figure}[h]
\begin{lstlisting}[style=csharpsmall,caption={Et eksempel på brug af Compass Sensor},label=kompas:kode_eksempel,frame=single]
HiTechnicCompassSensor compas =
	new HiTechnicCompassSensor();
brick.Sensor1 = compas;

compas.Poll();
Console.WriteLine(HiTechnicCompassSensor.Heading);
\end{lstlisting}
\end{figure}

\subsection{Præcisionstest}
Der blev udført i alt to tests; en med kompasset monteret på robotten og en hvor kompasset var del af en selvstændig konstruktion.
Test-værdier blev fundet ud fra samme metode, ved at rotere kompasset vilkårligt og derefter sammenligne vinklen mellem to afmålinger, ved brug af vinkelmåler.
I de to tabeller med oversigt over resultater ses først hvilke to kompas-afmålinger der sammenlignes, sammen med den målte vinkel.
Koden der blev kørt til udførelse af begge tests kan ses i \cref{kompas:kode_test}.

\begin{figure}[h]
\begin{lstlisting}[style=csharpsmall,caption={Kode brugt til test},label=kompas:kode_test,frame=single]
NxtBrick brick = new NxtBrick(NxtCommLinkType.Bluetooth, 8);
HiTechnicCompassSensor compas = new HiTechnicCompassSensor();
brick.Sensor1 = compas;

brick.Connect();

compas.OnPolled += pollable =>
	Console.WriteLine("Heading: {0}",
	(pollable as HiTechnicCompassSensor).Heading);
compas.PollInterval = 1000;

Console.ReadKey();

brick.Disconnect();
\end{lstlisting}
\end{figure}

\subsubsection{Test 1 - På robot}
Første forsøg blev udført med kompasset monteret ovenpå ultralyds-sensoren, for at holde en minimum-afstand på 15 cm fra brick og motorer.
Denne konstruktion var dog meget ustabil, da kompasset skulle være forholdsvist højt oppe ift. base-konstruktionen.
En oversigt over resultaterne er givet i \cref{kompas:resultat_paa_robot}.

\begin{table}[h]
\begin{tabularx}{\textwidth}{|>{\centering\arraybackslash}X|>{\centering\arraybackslash}X||>{\centering\arraybackslash}X|>{\centering\arraybackslash}X||>{\centering\arraybackslash}X|>{\centering\arraybackslash}X|}
\hline
\textbf{Aflæst} & \textbf{Målt} & \textbf{Aflæst} & \textbf{Målt} & \textbf{Aflæst} & \textbf{Målt} \\ \hline
0-2		& 2\dg& 44-60	& 12,5\dg& 172-194& 23\dg \\ \hline
2-4		& 1,5\dg& 60-72	& 13,5\dg& 194-224& 30,5\dg \\ \hline
4-6		& 1,5\dg& 72-84	& 11\dg& 224-256& 29,5\dg \\ \hline
6-8		& 1,5\dg& 84-100	& 16\dg& 256-282& 27,5\dg \\ \hline
8-8		& 2\dg& 100-108	& 9\dg& 282-354& 78\dg \\ \hline
8-10	& 1,5\dg& 108-110	& 0\dg& 354-356& 2\dg \\ \hline
10-16	& 6\dg& 110-116	& 7\dg& 356-358& 2,5\dg \\ \hline
16-26	& 11,5\dg& 116-144	& 24,5\dg& 358-358& 1,5\dg \\ \hline
26-38	& 10,5\dg& 144-158	& 11\dg& 358-0& 2\dg \\ \hline
38-44	& 8,5\dg& 158-172	& 10\dg& & \\ \hline
\end{tabularx}
\caption{Målinger foretaget med kompas på selvstændig og stabil konstruktion}
\label{kompas:resultat_stabil}
\end{table}

\begin{description}
\item[Største afvigelse]{8\dg}
\item[Mindste afvigelse]{0\dg}
\item[Gennemsnitlig afvigelse]{2.67}
\end{description}

\subsubsection{Test 2 - På stabil konstruktion}
Andet forsøg blev udført med kompasset monteret på en selvstændig og langt mere stabil konstruktion, for at se om dette ville kunne forbedre resultaterne.
Disse resultater kan ses i \cref{kompas:resultat_stabil}.

\begin{table}[h]
\begin{tabularx}{\textwidth}{|>{\centering\arraybackslash}X|>{\centering\arraybackslash}X||>{\centering\arraybackslash}X|>{\centering\arraybackslash}X||>{\centering\arraybackslash}X|>{\centering\arraybackslash}X|}
\hline
\textbf{Aflæst} & \textbf{Målt} & \textbf{Aflæst} & \textbf{Målt} & \textbf{Aflæst} & \textbf{Målt} \\ \hline
0-0		& 4,5\dg	& 110-120	& 12\dg		& 296-314	& 20,5\dg \\ \hline
0-2		& 0,5\dg	& 120-146 	& 18\dg		& 314-338	& 21\dg \\ \hline
2-14	& 13\dg		& 146-160	& 15,5\dg	& 338-338	& 6\dg \\ \hline
14-32	& 22\dg		& 160-188	& 20,5\dg	& 338-340	& 0\dg \\ \hline
32-42	& 11\dg		& 188-210	& 21\dg		& 340-358	& 17\dg \\ \hline
42-68	& 29\dg		& 210-234	& 22\dg		& 358-360		& 2,5\dg \\ \hline
68-74	& 5\dg		& 234-262	& 30,5\dg	& 			& \\ \hline
74-94	& 19\dg		& 262-288	& 26\dg		& 			& \\ \hline
94-110	& 13\dg		& 288-296	& 10,5\dg	& 			& \\ \hline
\end{tabularx}
\caption{Målinger foretaget med kompas monteret på stang over ultralyds-sensor}
\label{kompas:resultat_paa_robot}
\end{table}

\begin{description}
\item[Største afvigelse]{6\dg}
\item[Mindste afvigelse]{0\dg}
\item[Gennemsnitlig afvigelse]{1.66}
\end{description}
