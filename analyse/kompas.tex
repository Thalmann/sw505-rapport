\section{HiTechnic NXT Compass Sensor}
Kompasset fra HiTechnic er et digitalt kompas der måler jordens magnetiske felt.
Kompasset kan returnere en værdi der repræsenterer kompassets nuværende orientering.
Ifølge HiTechnic er værdierne returneret fra kompasset præcise ned til 1 grad.
Kalibrering skulle desuden ikke være nødvendig.
Dog er det nødvendigt for at minimere forstyrrelser, at kompasset holdes 10-15 cm væk fra forstyrrende elementer, herunder \lego NXT og motorer.\cite{hitechnic_compass}

I \cref{kompas:precision} undersøges hvorvidt målinger fra kompasset kan leve op til disse oplysninger.

\subsection{Compass Sensor i \mindsqualls}
Til at styre denne sensor i \mindsqualls, skal der bruges en af de seperate klasser til HiTechnic sensorerne, som findes i namespacet \lstinline[style=csharp]!NXT.MindSqualls.HiTechnic!.
Til kompas sensoren anvendes klassen \lstinline[style=csharp]!HiTechnicCompassSensor!.

\subsubsection{Aflæsning af værdier}
Kompas sensoren aflæses som alle andre sensorer, da den også arver fra \lstinline[style=csharp]!Pollable!-klassen.
Ved aflæsning af egenskaben \lstinline[style=csharp]!Heading!, gives et ulige heltal mellem 0 og 359.\footnote{Det har været nødvendigt at ændre implementationen af Heading for at aflæse kompassets værdi eksakt.
Se \cref{} for yderliger detaljer.\mikkel{Det skal beskrives hvilke ændringer der er lavet til klassen og hvorfor - jeg tænker at det bare skal i appendix.}}
Et eksempel på en simpel aflæsning af værdi kan ses i \cref{kompas:kode_eksempel}.

\begin{figure}[h]
\begin{lstlisting}[style=csharpsmall,caption={Et eksempel på brug af Compass Sensor},label=kompas:kode_eksempel,frame=single]
HiTechnicCompassSensor compas =
	new HiTechnicCompassSensor();
	
/* Instansiering af NxtBrick og forbindelse til NXT */

compas.Poll();
Console.WriteLine(compas.Heading);
\end{lstlisting}
\end{figure}

\subsection{Præcisionstest}\label{kompas:precision}
Der blev udført i alt to tests; en med kompasset monteret på robotten og en hvor kompasset var del af en selvstændig konstruktion.
Test-værdier blev fundet ud fra samme metode, ved at rotere kompasset vilkårligt og derefter sammenligne vinklen mellem to afmålinger, ved brug af vinkelmåler.
I de to tabeller, med oversigt over resultater, ses først hvilke to kompas-afmålinger der sammenlignes, sammen med den målte vinkel.
Yderligere er der lavet et boksplot for de to tests, hvor prikken repræsenterer gennemsnits-afvigelsen.
Koden der blev kørt til udførelse af begge tests kan ses i \cref{kompas:kode_test}.

\begin{figure}[h]
\begin{lstlisting}[style=csharpsmall,caption={Kode brugt til test},label=kompas:kode_test,frame=single]
NxtBrick brick = new NxtBrick(NxtCommLinkType.Bluetooth, 8);
HiTechnicCompassSensor compas = new HiTechnicCompassSensor();
brick.Sensor1 = compas;

brick.Connect();

compas.OnPolled += pollable =>
	Console.WriteLine("Heading: {0}",
	(pollable as HiTechnicCompassSensor).Heading);
compas.PollInterval = 1000;

Console.ReadKey();

brick.Disconnect();
\end{lstlisting}
\end{figure}

\subsubsection{Test 1 - På robot}
Første forsøg blev udført med kompasset monteret ovenpå ultralyds-sensoren, for at holde en minimum-afstand på 15 cm fra brick og motorer.
Denne konstruktion var dog meget ustabil, da kompasset skulle være forholdsvist højt oppe ift. base-konstruktionen.
En oversigt over resultaterne er givet i \cref{kompas:resultat_paa_robot}.

\begin{description}
\item[Største afvigelse]{8\dg}
\item[Mindste afvigelse]{0\dg}
\item[Gennemsnitlig afvigelse]{2.67}
\end{description}

\begin{table}[h]
\begin{tabularx}{\textwidth}{|>{\centering\arraybackslash}X|>{\centering\arraybackslash}X||>{\centering\arraybackslash}X|>{\centering\arraybackslash}X||>{\centering\arraybackslash}X|>{\centering\arraybackslash}X|}
\hline
\textbf{Aflæst} & \textbf{Målt} & \textbf{Aflæst} & \textbf{Målt} & \textbf{Aflæst} & \textbf{Målt} \\ \hline
0-2		& 2\dg& 44-60	& 12,5\dg& 172-194& 23\dg \\ \hline
2-4		& 1,5\dg& 60-72	& 13,5\dg& 194-224& 30,5\dg \\ \hline
4-6		& 1,5\dg& 72-84	& 11\dg& 224-256& 29,5\dg \\ \hline
6-8		& 1,5\dg& 84-100	& 16\dg& 256-282& 27,5\dg \\ \hline
8-8		& 2\dg& 100-108	& 9\dg& 282-354& 78\dg \\ \hline
8-10	& 1,5\dg& 108-110	& 0\dg& 354-356& 2\dg \\ \hline
10-16	& 6\dg& 110-116	& 7\dg& 356-358& 2,5\dg \\ \hline
16-26	& 11,5\dg& 116-144	& 24,5\dg& 358-358& 1,5\dg \\ \hline
26-38	& 10,5\dg& 144-158	& 11\dg& 358-0& 2\dg \\ \hline
38-44	& 8,5\dg& 158-172	& 10\dg& & \\ \hline
\end{tabularx}
\caption{Målinger foretaget med kompas på selvstændig og stabil konstruktion}
\label{kompas:resultat_stabil}
\end{table}

\begin{figure}[h]
\centering
\begin{tikzpicture}[thick]
\draw(1,0) rectangle (3,1); % Boks
\draw (2,0) -- (2,1); % Median streg
\draw (3,0.5) -- (8,0.5); % Fra øvre kvartil til maks
\draw (1,0.5) -- (0,0.5);% Fra nedre kvartil til min
\draw (8,0.3) -- (8,0.7); % Maksimum vertikal streg
\draw (0,0.3) -- (0,0.7); % Minimum vertikal streg
\filldraw[color=black] (2.67,0.5) circle (0.05cm); % Gennemsnittet

% Linje med værdier
\draw (0,-1) -- (10,-1);
\draw[snake=ticks,segment length=1cm] (0,-1) -- (10,-1);

% Tal på linje
\node[below] at (0,-1.1) {$0$};
\node[below] at (1,-1.1) {$1$};
\node[below] at (2,-1.1) {$2$};
\node[below] at (3,-1.1) {$3$};
\node[below] at (4,-1.1) {$1$};
\node[below] at (5,-1.1) {$2$};
\node[below] at (6,-1.1) {$3$};
\node[below] at (7,-1.1) {$3$};
\node[below] at (8,-1.1) {$8$};
\end{tikzpicture}
\caption{Boksplot for Test 1}
\label{kompas:test_1_boksplot}
\end{figure}

\subsubsection{Test 2 - På stabil konstruktion}
Andet forsøg blev udført med kompasset monteret på en selvstændig og langt mere stabil konstruktion, for at se om dette ville kunne forbedre resultaterne.
Disse resultater kan ses i \cref{kompas:resultat_stabil}.

\begin{description}
\item[Største afvigelse]{6\dg}
\item[Mindste afvigelse]{0\dg}
\item[Gennemsnitlig afvigelse]{1.66}
\end{description}

\begin{table}[h]
\begin{tabularx}{\textwidth}{|>{\centering\arraybackslash}X|>{\centering\arraybackslash}X||>{\centering\arraybackslash}X|>{\centering\arraybackslash}X||>{\centering\arraybackslash}X|>{\centering\arraybackslash}X|}
\hline
\textbf{Aflæst} & \textbf{Målt} & \textbf{Aflæst} & \textbf{Målt} & \textbf{Aflæst} & \textbf{Målt} \\ \hline
0-0		& 4,5\dg	& 110-120	& 12\dg		& 296-314	& 20,5\dg \\ \hline
0-2		& 0,5\dg	& 120-146 	& 18\dg		& 314-338	& 21\dg \\ \hline
2-14	& 13\dg		& 146-160	& 15,5\dg	& 338-338	& 6\dg \\ \hline
14-32	& 22\dg		& 160-188	& 20,5\dg	& 338-340	& 0\dg \\ \hline
32-42	& 11\dg		& 188-210	& 21\dg		& 340-358	& 17\dg \\ \hline
42-68	& 29\dg		& 210-234	& 22\dg		& 358-360		& 2,5\dg \\ \hline
68-74	& 5\dg		& 234-262	& 30,5\dg	& 			& \\ \hline
74-94	& 19\dg		& 262-288	& 26\dg		& 			& \\ \hline
94-110	& 13\dg		& 288-296	& 10,5\dg	& 			& \\ \hline
\end{tabularx}
\caption{Målinger foretaget med kompas monteret på stang over ultralyds-sensor}
\label{kompas:resultat_paa_robot}
\end{table}

\begin{figure}[h]
\centering
\begin{tikzpicture}[thick]
\draw(0.5,0) rectangle (2.5,1); % Boks
\draw (1.5,0) -- (1.5,1); % Median streg
\draw (2.5,0.5) -- (6,0.5); % Fra øvre kvartil til maks
\draw (0.5,0.5) -- (0,0.5);% Fra nedre kvartil til min
\draw (6,0.3) -- (6,0.7); % Maksimum vertikal streg
\draw (0,0.3) -- (0,0.7); % Minimum vertikal streg
\filldraw[color=black] (1.66,0.5) circle (0.05cm); % Gennemsnittet

% Linje med værdier
\draw (0,-1) -- (10,-1);
\draw[snake=ticks,segment length=1cm] (0,-1) -- (10,-1);

% Tal på linje
\node[below] at (0,-1.1) {$0$};
\node[below] at (1,-1.1) {$1$};
\node[below] at (2,-1.1) {$2$};
\node[below] at (3,-1.1) {$3$};
\node[below] at (8,-1.1) {$8$};
\end{tikzpicture}
\caption{Boksplot for Test 1}
\label{kompas:test_1_boksplot}
\end{figure}
