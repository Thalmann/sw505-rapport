Til at kortlægge rummet robotten befinder sig i, har vi valgt at bruge \textit{occupancy grid}.
Teorien bag denne algoritme vil blive beskrevet i dette afsnit.

\section{Overblik}
Den overordnede tanke bag et occupancy grid er, at lave en ensartet inddeling af sit kort, hvor hver enkelt celle er repræsenteret af en binær \textit{stokastisk variabel}, der fortæller om den pågældende celle er 'optaget' eller ej, hvor optaget betegnes som sandsynligheden $\mathcal{P}(occupied) = 1$.
Til at begynde med initialiseres hver enkelt celle med værdien $\mathcal{P}(occupied) = 0,5$ som en indikation på, at den aktuelle tilstand endnu ikke er kendt.

En 'ledig' celle har således værdien $\mathcal{P}(occupied) = 0$.
En simpel illustration af et occupancy grid map for det kørselsmiljø, der er opstillet for vores robot, kan ses på \cref{map:approx_occupancy_grid}.

\begin{figure}[h] % Kørselsmiljø og et occupancy grid
\centering
	\begin{subfigure}[b]{.45\textwidth}
	\centering
	\includegraphics[width=\textwidth]{verden/oppefra}
	\caption{Aktuelt Kørselsmiljø}
	\label{map:world}
	\end{subfigure}
	\begin{subfigure}[b]{.45\textwidth}
	\centering
	\includegraphics[width=\textwidth]{verden/occupancy_grid_verden}
	\caption{Eksempel på Occupancy Grid}
	\label{map:occupancy_grid}
	\end{subfigure}
\caption{Illustration af et occupancy grid baseret på projekts kørselsmiljø for robotten. Sorte celler i \cref{map:occupancy_grid} indikerer at $\mathcal{P}(occupied) = 1$, hvilket betegner væggene i kørselsmiljøet (\cref{map:world}). Hvide celler indikerer at $\mathcal{P}(occupied) = 0$ og grå celler angiver ikke-udforsket område. Den røde cirkel indikerer robottens position.}
\label{map:approx_occupancy_grid}
\end{figure}

\section{Vertikal og horisontal sensor model}\label{mapping:sensormodel}
Hvis vi antager, at alle objekter i opstillingen er placeret vinkelret til områdets x og y akser,
kan vi konstruere en forholdsvis simpel basal sensormodel.
Først beregner vi afstanden fra robotten til den pågældende celle i vores occupancy grid.

\begin{equation}
r = \mid x_{cell} - x_{robot} + y_{cell} - y_{robot} \mid
\end{equation}

Sensormodellen tildeler de celler, som ligger tæt på den målte afstand fra robotten $z_t$ en højere værdi, kaldet $l_{occ}$, end den prior \textit{belief} $l_0$ fra \cref{eqn:l0}.
De celler som ligger imellem robotten og sensormålingen, tildeles en lavere værdi end $l_0$, kaldet $l_{free}$. 

for at komme frem til hvad tæt på $z_t$ er, indfører vi en konstant $\alpha$ som repræsentere den gennemsnitslige tykkelse af objekter i området.

Sandsynligheden for at cellen er \emph{occupied}, kaldet $l_r$, tildeles således.

\begin{equation}
l_{r} = \begin{cases} 
	l_0 &\text{hvis }r > \text{min}(z_{max},z_t+\frac{\alpha}{2}) \\ 
	l_{occ} &\text{hvis } z_t-\frac{\alpha}{2} \leq r \leq z_t+\frac{\alpha}{2}\\ 
	l_{free} &\text{ellers}  
\end{cases}
\end{equation} \\
Hvor $z_{max}$ er sensorens maksimale måleafstand.
\\
Modellen vil se således ud beskrevet i pseudokode:

\begin{algorithm}[H]
\textbf{InversSensorModel($m_i, x_t, z_t$)} \\
Let $x_i,y_i$ be the center-of-mass of $m_i$ \\
$r = |x_i - x + y_i - y|$ \\
\If{$r > min(z_{max}, z_t + \frac{\alpha}{2}$)}{
	\Return{$l_0$}
}
\ElseIf{$z_t - \frac{\alpha}{2} \le r \le z + \frac{\alpha}{2}$}{
	\Return{$l_{occ}$}
}
\Else{
	\Return{$l_{free}$}
}
\caption{Invers sensor model algoritme.}
\label{alg:inversesensormodel}
\end{algorithm}

\subsection{Gaussisk sensor model}\label{mapping:gaussisk}

% forklaring af gaussisk støj (central limit theorem)
% tilfældige fejl vil tilnærme sig en gausssisk kurve.
I modellen, beskrevet i det forgående afsnit, antog vi, at en celle
med afstanden $\frac{\alpha}{2}$ fra robottens måling på $z_t$ havde
samme sandsynlighed som en celle med afstanden 0 fra $z_t$.

Hvis vi antager, at de celler, som ligger tættere på robottens måling, har en større sandsynlighed for at være \emph{occupied}, end dem som ligger tæt på $z_t \pm \frac{\alpha}{2}$, kan vi konstruere en sensor model med en glidende overgang fra værdien $l_{occ}$ for cellen i $z_t$ til værdierne $l_{free}$ og $l_0$ for cellerne på position $z_t \pm \frac{\alpha}{2}$. 

Da summen af uafhængige fejlmålinger, ifølge \emph{central limit theorem} vil tilnærme sig
den gaussiske normalfordeling, vil det være en god approximation for robottens måleusikkerhed.\cite[p. 223]{ArtificialIntelligence}

For at finde en passende normalfordeling skal vi vælge en passende middelværdi og en passende standard afvigelse. 
Vi ved at centrum, dvs. middelværdien, for fordelingen skal være målingen $z_t$.

\begin{figure}
\centering \includegraphics[scale=.75]{NormalDist}
\label{normaldistimg}
\caption{Normal fordeling $\mathcal{N}(\mu,\sigma^2)$}
\end{figure}

Udfra \cref{normaldistimg} kan vi se, at det vil være passende hvis $\frac{\alpha}{2}$ svarer til tre standard afvigelser. dvs.
\begin{equation}
	3\sigma = \frac{\alpha}{2} \implies \sigma = \frac{\alpha}{6}
\end{equation}

Vi kan anvende den passende normalfordeling $\mathcal{N}(z_t,\big(\frac{\alpha}{6}\big)^2)$ der ses her. 

\begin{equation}
\mathcal{N}\bigg(z_t,\bigg(\frac{\alpha}{6}\bigg)^2\bigg) = 
\frac{1}{\sqrt{2 \pi \big(\frac{\alpha}{6}\big)^2}}e^{- \frac{(x - z_t)^2}{2 (\frac{\alpha}{6})^2}}
\end{equation}

Vi kan beregne en sandsynlighed, for en celle i afstand $r$, udfra normalfordelingen ved at tage integralet af fordelingen på følgende måde.

\begin{equation}
P(r) = \lim \limits_{\rho \to 0} \bigint_{r-\rho}^{r+\rho} \frac{1}{\sqrt{2 \pi \big(\frac{\alpha}{6}\big)^2}}e^{- \frac{(x - z_t)^2}{2 (\frac{\alpha}{6})^2}}\, \mathrm{d}x
\end{equation}

% Indsæt billede som viser hvorden mapningen vil se ud.

Da vi vil have at sandsynlighederne mellem $z_t-\frac{\alpha}{2}$ og $z_t$ går i en glidende overgang fra $P_{free}$ til $P_{occ}$,
mens vi vil have en overgang fra $P_{occ}$ til ${P_0}$ for afstande imellem $z_t$ og $z_t+\frac{\alpha}{2}$.
Har vi valgt at lave en linær mapning, af de sandsynligheder vi får fra normalfordelingen til de to intervaller, ved hjælp af linjens ligning.
Her er $P_{occ}$,$P_{free}$ og $P_0$ sandsynlighederne for henholdsvis $l_{occ}$,$l_{free}$ og $l_0$.


\begin{equation}
	P_\kappa(r) = \begin{cases}
		\frac{P_{occ}-P_0}{P(z_t)-P(z_t+\frac{\alpha}{2})}(r-P(z_t))+P_{occ} &\text{hvis } z_t < r \\
		\frac{P_{free}-P_{occ}}{P(z_t-\frac{\alpha}{2})-P(z_t)}(r-P(z_t))+P_{occ} &\text{hvis } z_t \leq r 
	\end{cases}
\end{equation}

%######################################

%Da vi vil have, at sandsynligheden i centrum af fordelingen er den samme, som sandsynligheden fra $l_{occ}$ ganger vi med konstanten $\eta$, for på den måde at strække fordelingen så den passer. 

%\begin{equation}
%	P_\eta(r) = \eta P(r) 
%\end{equation}

%For at beregne $\eta$ udfra $l_{occ}$ benytter vi os af \cref{logodds:bel}.

%\begin{equation}
%	\eta P(r) = 1 - \frac{1}{1+e^{l_{occ}}} \implies \eta = \frac{1-\frac{1}{1+e^{l_{occ}}}}{P(r)}
%\end{equation}

%########################################

%\begin{equation}
%	\begin{split}
%		&\eta \lim \limits_{\rho \to 0} \bigint_{z_t-\rho}^{z_t+\rho} \frac{1}{\sqrt{2 \pi \big(\frac{\alpha}{6}\big)^2}}e^{- \frac{(x - z_t)^2}{2 (\frac{\alpha}{6})^2}}\, \mathrm{d}x = 1 - \frac{1}{1+e^{l_{occ}}} \\ 
%	&\implies \eta = \frac{1 - \frac{1}{1+e^{l_{occ}}}}{\lim \limits_{\rho \to 0} \bigint_{z_t-\rho}^{z_t+\rho} \frac{1}{\sqrt{2 \pi \big(\frac{\alpha}{6}\big)^2}}e^{- \frac{(x - z_t)^2}{2 (\frac{\alpha}{6})^2}}\, \mathrm{d}x}
%	\end{split}
%\end{equation}

Den nye tildeling af sandsynlighed til cellen vil nu se således ud.

%\begin{equation}
%	l_{r} = \begin{cases} 
%		l_0 &\text{hvis }r > \text{min}(z_{max},z_t+\frac{\alpha}{2}) \\
%		max\Bigg(l_{0},log\Big(\frac{P_\eta(r)}{1-P_\eta(r)}\Big)\Bigg) &\text{hvis } z_t < r \leq z_t+\frac{\alpha}{2} \\
%		max\Bigg(l_{free},log\Big(\frac{P_\eta(r)}{1-P_\eta(r)}\Big)\Bigg) &\text{hvis } z_t-\frac{\alpha}{2} \leq r \leq z_t \\
%		l_{free} &\text{ellers}	
%	\end{cases} 
%\end{equation}
%Hvor $l_0$ er defineret i \cref{eqn:l0} og $z_{max}$ er sensorens maksimale måleafstand.

\begin{equation}
	l_{r} = \begin{cases} 
		l_0 &\text{hvis }r > \text{min}(z_{max},z_t+\frac{\alpha}{2}) \\
		log\Big(\frac{P_\kappa(r)}{1-P_\kappa(r)}\Big) &\text{hvis } z_t < r \leq z_t+\frac{\alpha}{2} \\
		log\Big(\frac{P_\kappa(r)}{1-P_\kappa(r)}\Big) &\text{hvis } z_t-\frac{\alpha}{2} \leq r \leq z_t \\
		l_{free} &\text{ellers}	
	\end{cases} 
\end{equation}
Hvor $l_0$ er defineret i \cref{eqn:l0} og $z_{max}$ er sensorens maksimale måleafstand.



%
%\subsection{Generel udgave af sensor modellen}
%
%% objekter kan placeres i alle vinkler
%
%\subsubsection{Generel model med gaussisk støj}
%
%% gaussisk støj i 2-d
%
%
%\subsection{tilpasset sensor model}
%
%
%% overfitting
%
%\subsection{sensor model baseret på målinger}
%
%% fejl i sensor er ikke nødvendigvis 'tilfældige' eller uafhængige
%% en generel sensormodel som benytter en sandsynlighedsdistribution som vi selv har målt.
%
%
%\subsection{data udledet udfra robottens placering}
%
%% forbedring hvor vi tager højde for robotten
%
%
%\section{Forward Sensor Model	}
%
%
%














\
