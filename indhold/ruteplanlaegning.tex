\label{ruteplanleagning}
\newcommand{\unkcell}[3][]{\node (robot) [draw,fill=yellow, text centered, rectangle,
minimum height=\cellsize cm,minimum width=\cellsize cm, align=right] at ($(#2*\cellsize,#3*\cellsize) + (\cellsize/2,\cellsize/2)$) {$\scriptstyle #1$};}

\newcommand{\emptycell}[3][]{\node (robot) [draw,fill=green, text centered, rectangle,
minimum height=\cellsize cm,minimum width=\cellsize cm, align=right] at ($(#2*\cellsize,#3*\cellsize) + (\cellsize/2,\cellsize/2)$) {$\scriptstyle #1$};}

\newcommand{\occcell}[3][]{\node (robot) [draw,fill=red, text centered, rectangle,
minimum height=\cellsize cm,minimum width=\cellsize cm, align=right] at ($(#2*\cellsize,#3*\cellsize) + (\cellsize/2,\cellsize/2)$) {$\scriptstyle #1$};}

\newcommand{\scancell}[3][]{\node (robot) [draw,fill=orange, text centered, rectangle,
minimum height=\cellsize cm,minimum width=\cellsize cm, align=right] at ($(#2*\cellsize,#3*\cellsize) + (\cellsize/2,\cellsize/2)$) {$\scriptstyle #1$};}

\newcommand{\destcell}[3][]{\node (robot) [draw,fill=blue, text centered, rectangle,
minimum height=\cellsize cm,minimum width=\cellsize cm, align=right] at ($(#2*\cellsize,#3*\cellsize) + (\cellsize/2,\cellsize/2)$) {$\scriptstyle #1$};}

For at automatisere kortlægningen af et rum, er det nødvendigt at automatisere ruteplanlægningen.
Dette kapitel handler om den udledede metode til at planlægge en rute for robotten, så der bliver valgt den bedst mulige rute ud fra valgte kriterier.

\section{Overordnet beskrivelse}
Den overordnede idé er at efter robotten har taget en scanning, skal den ud fra de celler som har høj sandsynlighed for at være ledige, finde den næste lokation hvor der skal foretages en scanning.
Der gås ud fra at robotten opererer indenfor et occupancy grid, derved vil der for destinationer og målinger blive brugt celler.

\subsubsection{Destinationscelle}\label{rute:destinationscelle}
En mulig destinationscelle er en celle med høj sandsynlighed for at være ledig.
Desuden skal et vist antal celler omkring den mulige destinationscelle ligeledes være ledig, afhængig af robottens størrelse, for at være sikker på at der er plads til robotten.
Yderlige skal der være en rute hen til destinationscellen, hvor der er nok ledige celler i hele ruten, så robotten kan være indenfor de celler.

\subsubsection{Synlig celle}\label{rute:synligcelle}
De synlige celler for en destinationscelle, er de celler som vil blive opdateret ved en scanning fra destinationscellen.
Da vi opererer i en vinkelret verden, vil synsfeltet for en given destinationscelle være de celler som ligger vandret til højre og venstre for cellen, samt de celler som ligger lodret over og under cellen (se \cref{rute:synsfelt}).
Da sensoren(\cref{sensorer:us:resultater}) med sikkerhed har en maksimal rækkevidde på 170 cm, vil det også gælde her.
Dvs. at felter der har afstand > 170 cm ikke er en del af synsfeltet.

\subsubsection{Kriterier for en god destination}
Dette afhænger af den \textit{information gain} som robotten potentielt ville få ved at scanne i destinationen.
Dette afgøres af de synlige celler for destinationscellen.

Hvor høj \textit{information gain} der opnås, skal bestemmes ud fra sandsynlighederne der er i forvejen, for cellerne i synsfeltet.
Kort sagt er der bedst \textit{information gain} på en celle, hvor der er flest ikke-hidtil-scannede celler i synsfeltet.

\section{Beregning af næste målings-celle}
Til at beregne den næste destination, hvor fra der skal foretages en måling, skal følgende bruges:
\begin{itemize}
\item{$C = \{ c \mid c \textnormal{ er en celle i grid'et } \}$}
\item{$D = \{ c \mid c \textnormal{ er en destinationscelle}\}$}
\item{$x$}
\item{$p(d) = \{ c \mid c \textnormal{ er en celle i synsfeltet for } d \}$}
\end{itemize}
Hvor $C$ er mængden af alle celler der er i occupancy grid'et, $D$ er alle mulige destinationsceller, $x$ er den celle hvori robotten er og $p(d)$ er alle celler i synsfeltet for en destinationscelle $d$.
Se evt. \cref{rute:destinationscelle} for beskrivelse af destinationsceller og synlige celler.

\subsubsection{Beregning af \textit{information gain}}
Første skridt er at beregne værdi for alle de mulige destinationsceller:
\begin{equation}
v(d) = \sum_{c \in p(d)} 0.5-|0.5 - P(c)|
\end{equation}

Her er høj værdi lig med høj \textit{information gain}.
Dette skyldes at alle celler har sandsynlighedsværdien $0.5$ til at starte med, og når de opdateres kommer de enten tættere på $0$ eller $1$, afhængigt af om den er \textit{free} eller \textit{occupied}.
Derved er der mere værdi i at scanne fra en celle med lav værdi, da dette betyder at der er flere hidtil ikke-scannede celler i dens synsfelt.
Dette munder ud i en mængde, bestående af en værdi for samtlige $d$:
\begin{equation}
V = \{ v(d) \mid d \in D \}
\end{equation}

\subsubsection{Udvælgelse af den/de celler med lavest værdi}
\begin{equation}
Q = \{ d \in D \mid v(d) \in \min V \}
\end{equation}

\subsubsection{Beregning af afstand til robot}
\mikkel{Afstanden til robotten udregnes først - D findes vha. Dijkstra, og der findes afstandene også}
Hvis der i $Q$ er mere end én destinationscelle med samme værdi, beregnes afstanden til robotten for samtlige af de celler:
\begin{equation}
A = \{ \text{dist}(x,q) \mid q \in Q \}
\end{equation}

\subsubsection{Den endelige destinationscelle}
Til sidst findes den endelige destinationscelle, som enten er:
\begin{itemize}
\item{den med den laveste værdi.}
\item{den med kortest afstand til robotten, hvis der er flere med samme laveste værdi.}
\item{noget helt tredje hvis der slet ikke var mulige destinationsceller til at starte med.}
\end{itemize}
\begin{equation}
r = \begin{cases}
\begin{tabular}{lr}
$q \mid \text{dist}(x,q) \in \min A$ & $|Q| > 1$\\
$q \in Q$ & $|Q| = 1$ \\
$?$ & $|D| = 0$
\end{tabular}
\end{cases}
\end{equation}

\section{Algoritme}
\begin{algorithm}
\SetKwFunction{Fn}{GetNextScanDest}
\Fn{$D$}{\\
	
}
\end{algorithm}

\section{Illustration}

\begin{figure}
\centering
\begin{tikzpicture}[]
\newcommand{\gridsize}{10}
\newcommand{\cellsize}{0.7}

\coordinate (bl) at (0,0);
\coordinate (tr) at (\gridsize,\gridsize);

\foreach \x in {0,\cellsize,...,\gridsize}
	\foreach \y in {0,\cellsize,...,\gridsize}
		\draw [fill=yellow] ($(bl) + (\x,\y)$) rectangle ($(bl) + (\x + \cellsize, \y + \cellsize)$);
		
\emptycell{6}{6}
\emptycell{6}{7}
\emptycell{6}{8}
\emptycell{7}{6}
\emptycell[x]{7}{7}
\emptycell{7}{8}
\emptycell{8}{6}
\emptycell{8}{7}
\emptycell{8}{8}

\end{tikzpicture}
\label{rute:udgangspunkt}
\caption{Udgangspunkt}
\end{figure}
