\thilemann{Den øverste del er mere opsummering, hvor den sidste del er en forklaring af hvad vi gør}
\bruno{synes det er fint}
Denne del fokuserer på hvilke metoder og værktøjer der skal til for at løse problemstillingen for projektet.
Den beskriver de individuelle dele og deres formål og dermed også argumentationen for hvorfor netop det pågældende valg er truffet.

Dette afsnit har til formål at belyse sammenhængen af de enkelte dele; altså hvilken funktion de sammenhængende spiller i den tiltænkte implementering:
\\
\\
Da problemet består i at kortlægge en ukendt verden, vil der blive konstrueret en robot, der har til formål at navigerer i den pågældende verden via et indlejret system.
Dette vil blive implementeret på en NXT-enhed.
Systemet, som kører på robotten, skal skrives i programmeringssproget NXC, som er et C-lignende programmeringssprog.

For at robotten effektivt kan løse lokaliseringsproblemet vil den kommunikere med en PC via MindSqualls frameworket.
\thilemann{Bør vi skrive noget om at vi flytter de tunge beregninger til pc?}
PC'en skal være tilkoblet en Microsoft Kinect, og skal via colortracking med Kinectens farvekamera gøre det muligt at søge efter farver på robotten, og derved være i stand til at returnere en position til robotten, når den forespørger det.

Til at bygge kortet, i takt med at robotten navigerer rundt, benyttes et occupancy grid, der gør det muligt at differentiere mellem ledige og optagne celler.
Opdelingen af verdenen i celler, skal gøre det muligt for robotten enten at styrke eller mindske sin tro på, om et område på kortet er fremkommeligt eller ej.