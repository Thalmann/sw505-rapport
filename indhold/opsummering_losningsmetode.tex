Da problemet består i at kortlægge en ukendt verden, vil der blive konstrueret en robot, der har til formål at navigerer i den pågældende verden via et indlejret system.
Dette vil blive implementeret på en NXT-enhed.
Programmet, som kører på robotten, skal skrives i NXC.

Da det er computerdelen der skal stå for de tunge beregninger, vil robotten kommunikere med en computeren via MindSqualls.
PC'en skal være tilkoblet en Microsoft Kinect og skal via colortracking med Kinectens farvekamera gøre det muligt at søge efter farver på robotten, og derved være i stand til at sende en position til robotten, når den forespørger det.

Til at bygge kortet, i takt med at robotten navigerer rundt, benyttes occupancy grid, der gør det muligt at differentiere mellem ledige og optagne celler.
Opdelingen af verdenen i celler, skal gøre det muligt for robotten enten at styrke eller mindske sin tro på, om et område på kortet er fremkommeligt eller ej.
\mikael{Jeg synes dette kapitel er overflødigt. I dets nuværende form står der ikke mere information end i abstractet.}