% !TeX spellcheck = da_DK
Da baggrunden for projektet er fastlagt, defineres nu en problemformulering, der kort og præcist beskriver hvilket problem vi ønsker at løse.
Først afgrænses problemet, for på den måde at simplificerer det, og dermed gøre det realistisk at løse inden for tidsrammen.
Derefter præsenteres problemformuleringen, som projektet vil arbejde ud fra.
Til sidst vil en målsætning blive sat, så det er muligt at evaluere på resultatet til sidst i projektet.

\section*{Afgrænsning}
Pga. den stramme tidsramme foretages en række afgrænsninger.
For at undgå at skulle tage højde for en masse specialtilfælde antages det, at robotten altid tager sensormålinger når den står vinkelret på den bane, den kører på.
Desuden skal alle objekter i verdenen dermed også være vinkelrette. 
En anden afgrænsning er, at robottens verden er inde i et afgrænset område, hvilket gør lokaliseringen af robotten mere simpel.
Desuden afgrænses robottens verden til at være plan og indendøre. 
Dette gør kravene til konstruktionen af robotten simple, så fokus kan holdes på at lave en algoritme til mapping.

\paragraph{}
 Afgrænsningerne er således:
\begin{itemize}
\item Robottens verden er 90 grader
\item Verdenen er et afgrænset område
\item Verdenen er plan og befinder sig indendøre
\end{itemize}

\section*{Problemformulering}\label{problemformulering}
Vi vil arbejde ud fra følgende problemstilling på baggrund af ovenstående afgrænsning:

\begin{samepage}

\quoter{Hvordan kan der konstrueres software til en robot, hvis formål er at kortlægge en ukendt verden, forudsat at den til enhver tid kender sin position?}

\end{samepage}

\section{Målsætning}\label{problem:maalsaetning}
% !TeX spellcheck = da_DK
For at kunne evaluere på det færdige produkt, er det nødvendigt at have en konkret målsætning, der kan sammenlignes på. 

I forhold til dette projekt, er der forskellige muligheder.
Man kan forsøge at kortlægge et rum på kortest mulig tid, eller man kan forsøge at kortlægge rummet så præcist som muligt.

\begin{itemize}
\item Hastigheden af kortlægningen vil afhænge af robotkonstruktionen, sensorernes hastighed og algoritmernes kompleksitet.
\item Præcisionen af kortlægningen vil afhænge af sensorernes præcision og algoritmernes præcision.
\end{itemize}
Det er i projektgruppen besluttet at prioritere præcisionen af kortlægningen højest, hvilket udtrykkes i denne målsætning.

\paragraph{}
\noindent Målet med dette projekt er følgende:
\begin{itemize}
\item At bygge en robot, der kan konstruere et præcist kort.
\end{itemize}