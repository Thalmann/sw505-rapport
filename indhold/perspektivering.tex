% !TeX spellcheck = da_DK
Projektets resultater kan som altid forbedres.
Nogle forbedringer er bevidst ikke implementeret på grund af tidsbegrænsning. 
Andre forbedringer er gruppen kommet frem til i løbet af projektet.

I dette afsnit præsenteres forslag til mulige forbedringer af systemet, samt en vurdering af, hvor anvendeligt det endelige produkt er.

\section{Implementering}
I dette projekt, er der blevet implementeret to forskellige sensormodeller. 
Udfra de test vi har udført (se \cref{evaluering:testresultater}), kan vi se at den gaussiske sensormodel er bedre end den simple. \\
En mulig forbedring af den gaussiske model vil være at lave fordelingen i 2 dimensioner, således at alle de omkringliggende celler bliver opdateret. 
Dette giver mening, da der er en forøget sandsynlighed for, at en celle er optaget, hvis de omkringliggende celler er optaget. \\
Vi overvejede også at benytte en sensormodel som var baseret på vores målinger fra \cref{subsection:ultrasonic}.
Hvilket ville betyde at fordelingen reflektere sensorens faktiske måde at opføre sig på.

I de udførte test, er der benyttet en cellestørrelse på 10cm $ \times $ 10cm.
Denne cellestørrelse er blevet valgt med henblik på at gøre cellerne så små som muligt, uden at tiden det vil tage at kortlægge området, vil blive alt for stor.
Man kan således forsøge at variere størrelsen af celler for at finde ud af, hvilken indvirkning det har på resultaterne.

Allerede i problemformuleringen blev robottens verden afgrænset til kun at være 90 grader. 
Denne afgrænsning gjorde nogle aspekter af problemerne lettere. 
Blandt andet simplificerede det den måde sensormålinger skal behandles, når occupancy grid'et skal opdateres.
En forbedring vil være at ophæve denne begrænsning, så der kan tages målinger i 360 grader.

Ved test af systemet blev det bemærket, at der blev brugt en del tid på at robotten flyttede sig fra et punkt til et andet.
I dette tidsrum blev der ikke foretaget nogle målinger og var derfor 'spildt' tid. 
Dette kan forbedres ved at robotten løbende tager målinger, så tiden brugt mellem to punkter bliver mere effektivt udnyttet.

\section{Robotkonstruktion}
Robotten er blevet konstrueret med to sensorer der kan dreje. 
Således er det muligt at tage en sensormåling forud og bagud, og derefter dreje sensortårnet til at tage sensormålinger til begge sider.
Dette har vist sig at være unødvendigt, da tiden det tager at dreje tårnet gør det upraktisk.
Det viste sig også at komplicere registreringen af sensormålinger, da sensortårnet ikke kunne placeres lige over det punkt, der blev brugt til at lokalisere robotten.
En ændring af dette kunne være at fastmontere sensorerne og nøjes med at dreje robotten.
Et andet alternativ vil være at montere 4 sensorer i stedet for to, så det slet ikke vil være nødvendigt at dreje hverken et sensortårn eller robotten for at tage målingerne.

Ved konstruktionen af systemet blev det antaget, at ruteplanlægningen skulle have ansvaret for, at robotten ikke skulle køre ind i noget. 
Dette gøres ud fra det occupancy grid, der indtil videre er konstrueret.
Det viste sig dog til nogle af de indledende prøvekørsler, at dette ikke altid var tilstrækkeligt, og at robotten kunne finde på at køre ind i en væg eller en forhindring.
Det vil derfor være nyttigt at implementere en mekanisme, der identificerer, at robotten er på vej til at køre ind i en forhindring, og i stedet stoppe robotten inden det sker. 

\section{Anvendelse}
Systemet er afhængigt af, at der sidder et kamera i loftet og lokaliserer robotten.
Dette begrænser anvendeligheden en del, da det i mange tilfælde ikke vil være muligt at fastmontere et kamera i loftet.
Systemet kan således ikke benyttes til at kortlægge et rum der er ufremkommeligt, såsom i en krigszone eller på Mars.
Det vil derimod være muligt at bruge den til at kortlægge bygninger, som robotter efterfølgende skal navigere rundt i.
For at robotten skal være mere anvendelig, vil det være nødvendigt med en anden ekstern kilde til lokalisering, for eksempel GPS eller Wifi.
