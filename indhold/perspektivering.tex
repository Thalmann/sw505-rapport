% !TeX spellcheck = da_DK
Projektets resultater kan som altid forbedres.
Nogle elementer er bevidst ikke implementeret på grund af tidsbegrænsning, og andre er gruppen kommet frem til, givet de erfaringer der er gjort i løbet af projektet.

I dette afsnit præsenteres forslag til forbedringer af systemet der kunne gøres, samt en beskrivelse af anvendeligheden af det endelige produkt.

\section{Lokalisering}
Kinecten brugt til lokalisering fra starten af projektet, da det var meningen at bruge dybdesensoren til at lokalisere objekter i et billede.
Siden da blev løsningen ændret til at genkende objekter ved farve, men Kinecten blev beholdt.
Det ville derfor være muligt at udskifte Kinecten med et almindeligt webcam, som ville være lettere at komme i besiddelse af. 
Der er også et større udvalg af webcams, så det vil være muligt at finde en bedre opløsning, opdateringshastighed eller en anden parameter der vil forbedre lokaliseringen.

\section{Implementering}
I dette projekt er der blevet implementeret to forskellige sensormodeller. \stefan{indsæt vurdering af de to modeller}
En forbedring ville være at implementere endnu bedre sensormodeller.

I de udførte test er der benyttet en cellestørrelse på 10cm $ \times $ 10cm.
Denne cellestørrelse blev valgt med henblik på at få cellerne så små som muligt, uden at tiden det ville tage at kortlægge området ville blive alt for stor.
Man kunne således forsøge at variere størrelsen af celler for at finde ud af hvilken indvirkning det har på resultaterne.

Allerede i problemformuleringen blev robottens verden afgrænset til kun at være 90 grader. 
Denne afgrænsning gjorde nogle aspekter af problemer lettere. 
Blandt andet simplificerede det den måde sensormålinger skal behandles når occupancy griddet skal opdateres.
En forbedring ville være at ophæve denne begrænsning så der kan tages målinger i 360 grader.

Ved test af systemet blev det bemærket at der blev brugt en del tid på at robotten flyttede sig fra et punkt til et andet.
I dette tidsrum blev der ikke foretaget nogle målinger og var derfor 'spildt' tid. 
Dette kunne forbedres ved at robotten tog løbende målinger, så tiden brugt mellem to punkter blev mere effektivt udnyttet.

\section{Robotkonstruktion}
Robotten er blevet konstrueret med to sensorer der kan dreje. 
Således er det muligt at tage en sensormåling forud og bagud, og derefter dreje sensortårnet til at tage sensormålinger til begge sider.
Dette har vist sig at være unødvendigt, da tiden det tager at dreje tårnet gør det upraktisk.
Det viste sig også at komplicere registreringen af sensormålinger, da sensortårnet ikke kunne placeres lige over det punkt der blev brugt til at lokalisere robotten.
En ændring af dette kunne være at fastmontere sensorerne og nøjes med at dreje robotten.
Et andet alternativ vil være at montere 4 sensorer i stedet for to, så det slet ikke vil være nødvendigt at dreje hverken et sensortårn eller robotten for at tage målingerne.

Ved konstruktionen af systemet blev det antaget at ruteplanlægningen skulle have ansvaret for at robotten ikke kører ind i noget. 
Dette gøres ud fra det occupancy grid der ind til videre er konstrueret.
Det viste sig dog til nogle af de indledende prøvekørsler at dette ikke altid var tilstrækkeligt, og at robotten kunne finde på at køre ind i en væg eller en forhindring.
Det vil derfor være nyttigt at implementere en mekanisme der identificerer at robotten er på vej til at køre ind i en forhindring og i stedet stoppe robotten inden det sker. 

\section{Anvendelse}
Systemet er afhængig af at der sidder et kamera i loftet og lokaliserer robotten.
Dette begrænser anvendeligheden en del, da det i mange tilfælde ikke vil være muligt at fastmontere et kamera i loftet.
Systemet kan således ikke benyttes til at kortlægge et rum der er ufremkommeligt, såsom i en krigszone eller på Mars.
Det vil derimod være muligt at bruge den til at kortlægge en bygninger som robotter skal navigere rundt i efterfølgende.

For at robotten skulle være mere anvendelig, vil det være nødvendigt med en anden ekstern kilde til lokalisering.