Når en robot skal kortlægge et område er der to problemer. 
Det første er at robotten ikke ved hvor den befinder sig, dvs. den har ikke nogen lokation.
Det andet problem er at den ikke har et kort, fordi det er jo det den skal lave.
Dette problem er beskrevet i \cref{SLAM}

\section*{SLAM}\label{SLAM}
SLAM(Simultaneous Localization And Mapping) opstår når man skal kortlægge et område uden at kende robottens lokation.
De to problemer er afhængige af hinanden fordi robotten skal vide hvor den er henne for at lave et og vice versa.
Robotten bruger dens sensorer til at finde ud af hvad der er omkring.
Den bruger så motorer til at bevæge sig og kan på den måde finde ud af hvor den er.
Der er dog et problem, usikkerheden på sensorer og motor gør at robotten ikke nødvendigvis er der hvor den tror den er.
Hvis der fx er en usikkerhed på 3 grader og vi forestiller os at motoren har kørt frem 3 gange - så har vi en usikkerhed på +- 9 grader, hvilket hurtigt løber.
\\
Da dette problem er ret komplekst at løse vil vi kigge på problemet hvor vi kender lokationen beskrevet i \cref{map_lok}

\section*{Mapping med kendt lokation}\label{map_lok}
Når vi kender lokationen bliver ovenstående problem mindre komplekst.
Når robotten kender dens lokation skal den 'kun' kortlægge det ukendte område.
Det fungerer ved at robotten bruger sensorerne til at finde ud af hvordan omgivelserne ser ud og så kender sin nøjagtige lokation vha. fx GPS.
Det gør at der ikke opstår samme problem som beskrevet ovenover.