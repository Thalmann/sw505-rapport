Når en robot skal kortlægge et område er der overordnet set to problemer. 
Det første er at robotten ikke ved hvor den befinder sig, dvs. den har ikke nogen lokation.
Det andet problem er at den ikke har et kort over steder den har været.
Begge problemer set som ét problem er kendt som SLAM, hvilket er beskrevet i det følgende afsnit.

\section*{SLAM}\label{SLAM}
SLAM (Simultaneous Localization and Mapping) opstår når man skal kortlægge et område uden at kende robottens lokation.
De to problemer er afhængige af hinanden, da robotten skal vide hvor den er henne for at lave et kort og for at vide hvor den er skal den bruge et kort.
Uden noget eksternt der kan fortælle robotten hvor den er, er den nødsaget til at bruge dens sensorer til at finde ud af hvad der er omkring.
Den bruger samtidig motorer til at bevæge sig, hvilket kan bruges til at give den relative lokation i forhold til tidligere lokationer.
Der er dog et problem, usikkerheden på sensorer og motorer gør at robotten ikke nødvendigvis er der hvor den tror den er.
Samtidig kan der også være usikkerheder i de sensorer der benyttes, som fx. kompas, sonar, laser.

Da dette problem er ret komplekst at løse, vil vi kigge på problemet med kortlægning, hvor lokationen er kendt, som beskrevet i det følgende afsnit.

\section*{Mapping med kendt lokation}\label{map_lok}
Når vi kender lokationen bliver ovenstående problem mindre komplekst.
Når robotten kender dens lokation skal den blot kortlægge det ukendte område.
Det fungerer ved at robotten bruger sensorerne til at finde ud af hvordan omgivelserne ser ud og så kender sin nøjagtige lokation vha. fx. GPS.
Dette burde gerne fjerne de unøjagtigheder der er ift. robottens lokation.
