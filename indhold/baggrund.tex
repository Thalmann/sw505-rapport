Når en robot skal kortlægge et område er der overordnet set to problemer. 
Det første er at robotten ikke ved hvor den befinder sig, dvs. den har ikke nogen lokation.
Det andet problem er at den ikke har et kort over steder den har været.
Begge problemer set som ét problem er kendt som SLAM, hvilket er beskrevet i det følgende afsnit.

\section*{SLAM}\label{SLAM}
SLAM (Simultaneous Localization and Mapping) opstår når man skal kortlægge et område uden at kende robottens lokation.
De to problemer er afhængige af hinanden eftersom robotten skal kende sin lokation for at lave et kort, og for at kende dens lokation skal den bruge et kort.
Uden brug af ekstern udstyr der kan fortælle robotten hvor den befinder sig, er den nødsaget til at bruge dens sensorer til at finde ud af hvad dens omgivelser består af.
Den bruger samtidig dens motorer til at bevæge sig, hvilket kan bruges til at give den relative lokation i forhold til tidligere lokationer.
Der er dog et problem; usikkerheden på sensorer og motorer gør at robotten ikke nødvendigvis befinder sig hvor den faktisk tror den er.
Samtidig kan der også være usikkerheder i de sensorer der benyttes, som fx. kompas, sonar og laser.

Da dette problem er ret komplekst at løse \cite[s.~514]{thrun2002particle}, vil vi kigge på problemet med kortlægning, hvor lokationen er kendt, som beskrevet i det følgende afsnit.

\section*{Mapping med kendt lokation}\label{map_lok}
Når vi kender lokationen bliver ovenstående problem mindre komplekst.
I tilfælde hvor robotten kender sin lokation, skal den blot kortlægge det ukendte område.
Kortlægningen fungerer ved at robotten bruger dens sensorer til at finde ud af hvordan omgivelserne ser ud, og finder så sin umiddelbare lokation vha. f.eks. GPS.

Denne fremgangsmåde fjerner mange af de reducere der opstår når der fokuseres på SLAM problemet, og simplificerer således processen for at finde robottens lokation.