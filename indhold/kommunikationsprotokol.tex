\section{Kommunikationsprotokol}
Dette afsnit beskriver kommunikationen mellem NXT og PC.
Afsnitet beskriver de beskeder der bliver snedt frem og tilbage og hvilken encoding der bliver brugt.

\subsection{Besked typer}
Dette afsnit er delt op i to under afsnit som beskriver de forskellige beskeder fra PC til NXT \cref{design:pctilnxt} og fra NXT til PC \cref{design:nxttilpc}.
\subsubsection{PC til NXT}\label{design:pctilnxt}

\subsubsection{NXT til PC}\label{design:nxttilpc}

\begin{table}[h]
\label{design:protokol_tabel}
\caption{Overigt over de forskellige encodings og hvilke beskeder de indeholder.}
\begin{longtable}{ c | c | c | p{0.6\textwidth}}
Encoding & PC & NXT & Indhold\\
\hline
\hline
0 & IN & OUT & Robotten forespørger om dens lokation \\
1 & IN & OUT & Robotten er nået frem til dens lokation \\
2 & IN & OUT & Robotten sender sensordata \\
50 & OUT & IN & Computeren fortæller roboten at den skal bevæge sig til en position som sendes med\\
51 & OUT & IN & Computeren beder om at få sensor data fra robotten \\
52 & OUT & IN & Computeren sender robottens position til robotten \\
\end{longtable}
\end{table}

\subsection{title}