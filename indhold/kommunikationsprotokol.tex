% !TeX spellcheck = da_DK

Dette afsnit beskriver kommunikationen mellem NXT og computer.
Afsnittet beskriver de beskeder der bliver sendt frem og tilbage og de indkodes inden de sendes.

\subsection{Besked}
Beskeder vil i dette afsnit beskrives ved brug af vinklede parenteser til at indikere dele af beskeden.
En besked er opbygget på følgende måde:
\begin{equation}
<BeskedType><Indhold>
\end{equation}
Beskedens type er en eller to bytes.
Beskeder på én byte sendes fra robotten og modtages af computeren mens beskeder på to byte sendes fra computeren og modtages af robotten.
Beskedindholdet består af ingen eller flere bytes.
I det næste afsnit vil de forskellige beskeder der benyttes blive beskrevet.

\subsection{Besked typer}
Alle beskederne kan ses i \cref{design:protokol_tabel}.
\\
Første kolonne fortæller hvilken talkode der benyttes til at indikere beskedens type.
Anden og tredje kolonne fortæller henholdsvis om beskeden går fra computerens eller NXT'ens inbox eller outbox.
De to sidste beskriver hvad beskeden indeholder og gør.
Nedenfor er de forskellige scenarier som beskederne fra tabellen bliver brugt.


\begin{figure}[H]
\renewcommand{\arraystretch}{1.8}
\begin{longtable}{ c | c | c | p{0.4\textwidth} | c}
Besked type & Computer & NXT & Beskrivelse & Indhold\\
\hline
\hline
0 & IN & OUT & Robotten forespørger om dens lokation & - \\
1 & IN & OUT & Robotten er nået frem til dens lokation & - \\
2 & IN & OUT & Robotten sender sensordata & $s1_x,s2_x,s1_y,s2_y$ \\
50 & OUT & IN & Computeren fortæller roboten at den skal bevæge sig til en position  & $x,y$\\
51 & OUT & IN & Computeren beder om at få sensor data fra robotten & - \\
52 & OUT & IN & Computeren sender robottens position til robotten & $x,y,a$\\
\hline
\caption{Oversigt over de forskellige beskeder}\label{design:protokol_tabel}\\
\end{longtable}

\end{figure}

\subsubsection{Nuværende lokation}
Denne situation forekommer når robotten skal bede om sin lokation fra computeren.
Robotten sender en besked på formen
\[<0>\]
Derefter returner computeren en besked på formen
\[<52><x><y><a>\]
Denne besked indeholder lokationen og dens vinkel hvor $x$ og $y$ repræsenterer lokationen og $a$ vinkelen.

\subsubsection{Naviger til en ny lokation}\label{design:protokol_navigertilny}
Denne situation opstår når computeren sender en besked der fortæller robotten hvor den skal køre hen.
computerens besked er på formen
 \[<50><x><y>\] 
hvor $x$ og $y$ er koordinaterne til det punkt robotten skal køre hen til.

\subsubsection{Nået frem til lokation}
Denne situation forekommer når robotten er nået frem til den ønskede lokation.
Robotten sender da en besked på formen \[<1>\]

\subsubsection{Aflæs Sensordata}
Når computeren beder om sensor data fra robotten, sender den en besked på formen \[<51>\]
Robotten foretager derefter sensormålinger og returnerer dem med en besked på formen  
\[<2><s1_x> <s2_x> <s1_y> <s2_y>\]
Sensordata består af en måling fra hver sensor før sensorerne roterer 90 grader og efter.
$ s1_x $ og $ s2_x $ er derfor to målinger mens sensorerne står i startpositionen, mens $ s1_y $ og $ s2_y $ er to måling efter sensorerne er drejet 90 grader.