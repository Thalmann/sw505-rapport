%Robot vil vide hvor den er:
%\begin{enumerate}
%\item Robot lægger besked OUTGOING\_WHERE\_AM\_I i outbox
%\item Robotten går i gang med kontinuerligt at checke sin inbox efter beskeden INCOMING\_GET\_POS
%\begin{enumerate}
%\item Hvis beskeden er der:\\
%Opdater robottens pose med den der ligger i inbox
%\item Ellers:\\
%Check efter besked
%\end{enumerate}
%\item RobotInterface beder kontinuerligt Postman om at checke inbox (Robottens outbox)
%\begin{enumerate}
%\item Hvis der er besked og denne er RobotRequestsLocation:\\
%Bed LocationControl om robottens pose
%\item Hvis der er besked og den ikke er RobotRequestsLocation:\\
%Udfør den anden handling
%\item Ellers:\\
%Check efter besked
%\end{enumerate}
%\item LocationControl beder RobotLocation i TrackingServices om robottens pose
%\item RobotInterface beder Postman om at lægge robottens pose i outbox (Robottens inbox)
%\item Der var besked i robottens inbox
%\end{enumerate}

\section{Flow}
For at vise hvordan arkitekturen benyttes, og samtidig vise kommunikationen mellem robot og PC, kommer der her et udvalgt eksempel, hvor det kan ses hvordan data traverserer gennem lagene i arkitekturen.

\subsection{Eksempel: Robotten vil vide hvor den er}
Det overordnede flow kan ses i tilstand-sekvens-diagrammet i \cref{flow:ssd}, dog skal det bemærkes at en vis del af det egentlige data flow er abstraheret væk.
Den egentlige data flow vil blive beskrevet herefter, med kode-eksempler.

\begin{figure}[h]
\centering
\includegraphics[width=\textwidth]{arkitektur/ssd}
\caption{Tilstand-sekvens-diagram}
\label{flow:ssd}
\end{figure}

\subsubsection{Robotten sender anmodningen}
Det første der sker er at robotten lægger beskeden \lstinline[style=c]!OUTGOING_WHERE_AM_I! i \lstinline[style=c]!OUTBOX!:

\begin{lstlisting}[style=csmall,label=lst:whereami_request,caption=Robotten sender anmodning om positur]
string msg = NumToStr(OUTGOING_WHERE_AM_I);
SendMessage(OUTBOX, msg);
\end{lstlisting}

\subsubsection{Robotten venter på svar}
Robotten går i gang med kontinuerligt at checke \lstinline[style=c]!INBOX! indtil beskeden med typen \lstinline[style=c]!INCOMING_GET_POS! er at finde:

\begin{lstlisting}[style=csmall,label=lst:whereami_response,caption=Robotten venter på svar]
while(true) {
   ReceiveRemoteString(INBOX, true, in);
   if (StrLen(in) > 0) {
      cmd = SubStr(in, 0, 2);
      cmdType = StrToNum(cmd);
      if (cmdType == INCOMING_GET_POS) {...}
   }
}
\end{lstlisting}

\subsubsection{RobotInterface venter på besked fra robotten}
Håndtering af robottens beskeder/anmodninger sker i \lstinline[style=csharp]!RobotInterface!.
Her kører en seperat tråd med funktionen \lstinline[style=csharp]!listener()!, som sættes i gang når PC programmet startes.
Her håndteres blandt andet beskeden \lstinline[style=csharp]!NXTMessageType.RobotRequestsLocation!, som hentes fra \lstinline[style=csharp]!NXTPostman! ved kontinuerligt at checke efter nye beskeder.

\begin{lstlisting}[style=csharpsmall,label=lst:listener,caption=listener() i RobotInterface]
private void listener()
{
    while (RUNNING)
    {
        if (checkForMessages()) (*\label{listener:checkformessages}*) 
        {
            NXTMessage msg = postman.RetrieveMessage();

            switch (msg.MessageType) {...}
        }
        Thread.Sleep(THREAD_SLEEP_INTERVAL_IN_MILLISECONDS);
    }
}
\end{lstlisting}

I \cref{listener:checkformessages} i \cref{lst:listener} kontrolleres det om der findes besked i \lstinline[style=csharp]!PC_INBOX! af de indgående besked-typer:

\begin{lstlisting}[style=csharpsmall,label=lst:checkformessages]
private bool checkForMessages()
{
    return postman.HasMessageArrived(NXTMessageType.RobotRequestsLocation)
        || postman.HasMessageArrived(NXTMessageType.RobotHasArrivedAtDestination);
}
\end{lstlisting}

Hvis der findes besked, hentes denne via \lstinline[style=csharp]!retrieveMessage()! i \lstinline[style=csharp]!NXTPostman!, som bruger Bluetooth-forbindelsen til at kontrollere for nye beskeder på i robottens \lstinline[style=c]!OUTBOX!:

\begin{lstlisting}[style=csharpsmall,label=lst:postman,caption=NXTPostman henter besked fra robot]
public NXTMessage RetrieveMessage()
{
    try
    {
        byte[] msg = CommunicationBrick.CommLink.MessageReadToBytes(PC_INBOX, NxtMailbox.Box0, true);
        return new NXTMessage(msg);
    }
    catch {...}
}
\end{lstlisting}

Når beskeden \lstinline[style=csharp]!NXTMessageType.RobotRequestsLocation! findes i \lstinline[style=csharp]!PC_INBOX! udføres den pågældende metode \lstinline[style=csharp]!RobotRequestLocation()!.

%\begin{lstlisting}[style=csharpsmall,label=lst:robotrequestslocation,caption=RobotRequestsLocation() i RobotInterface]
%private void RobotRequestLocation()
%{
%    IPose pose = locCon.RobotPose;
%    string encodedMsg = NXTEncoder.Encode(pose);
%    byte[] byteEncMsg = NXTEncoder.ByteEncode(pose);
%    NXTMessage outMsg = new NXTMessage(NXTMessageType.SendPostion,
%        encodedMsg, byteEncMsg);
%    postman.SendMessage(outMsg);
%}
%\end{lstlisting}

\subsubsection{RobotInterface henter positur i LocationControl}
RobotInterface får den senest opdateret positur ved at aflæse en property på \lstinline[style=csharp]!LocationControl!.

\subsubsection{LocationControl henter positur i TrackingServices}
LocationControl får den senest opdateret positur ved at aflæse en property i \lstinline[style=csharp]!TrackingServices!.
\lstinline[style=csharp]!TrackingServices! opdaterer sin nuværende viden om robottens positur løbende, ved kontinuerligt at beregne en ny ud fra Kinect'ens billede.

\subsubsection{RobotInterface får NXTPostman til at sende posituren}
RobotInterface har nu det seneste data om robottens positur.
Den bruger så \lstinline[style=csharp]!NXTPostman! til at lægge beskeden af type \lstinline[style=csharp]!NXTMessageType.SendPosition! i \lstinline[style=csharp]!PC_OUTBOX!, som er robottens \lstinline[style=c]!INBOX!.

\subsubsection{Robotten modtager beskeden}
Robotten, som ventede på besked ved kontinuerligt at checke sin \lstinline[style=c]!INBOX! har nu fået en besked af typen \lstinline[style=c]!INCOMING_GET_POS! og kan derved opdatere sin interne repræsentation af dens nuværende positur:

\begin{lstlisting}[style=csmall,label=lst:updatepose,caption=Robotten opdaterer sin positur ud fra den modtaget fra PC]
if (cmdType == INCOMING_GET_POS) {
   string x = SubStr(in, 2, 7);
   string y = SubStr(in, 9, 7);
   string angle = SubStr(in, 16, 7);

   Pose po;
   po.p.x = StrToNum(x);
   po.p.y = StrToNum(y);
   po.angle = StrToNum(angle);

   return po;
}
\end{lstlisting}
