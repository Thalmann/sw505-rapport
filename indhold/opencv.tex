\section{OpenCV}
% evt. en generel beskrivelse af emgu og openCV
OpenCV (\emph{Open Source Computer Vision Library}) er et Library udviklet af \emph{Intel}
til brug ved Realtids \emph{computer vision}.
OpenCV er skrevet i C++ og er frit tilg�ngeligt under BSD licensen.

\subsection{beskrivelse af udvalgte klasser og funktioner}

\subsubsection{Metoden findContours}
Metoden findCountours gennems�ger et billede og finder konturer, som den kan returnere 
repr�senteret p� forskellige m�der, alt efter hvilke parametre metoden kaldes med.

De typiske repr�sentationer er CV\_RETR\_EXTERNAL som kun returnere de yderste konturer,
CV\_RETR\_LIST der returnere alle konturer som en liste uden nogen repr�sentation af deres hirarki og
CV\_RETR\_TREE som placere konturene i en tr�struktur der repr�sentere deres hirarki.

Metoden kan benytte sig af flere forskellige metoder til at overs�tte konturerne til en k�de af punkter. 
Der er f.eks. CV\_CHAIN\_APROX\_NONE som overs�tter konturen til punkter uden nogen form for komprimering.
En anden er CV\_CHAIN\_APROX\_SIMPLE der kun indeholder start og slutpunkter for alle horisontale,
vertikale og diagonale linjestykker, hvilket reducere antallet af punkter og derved algoritmens
pladskompleksitet.
\cite{EmguCVLibDoc}

\subsubsection{Metoden thresholdBinary}
Metoden tresholdBinary tager to parametre som input en threshld v�rdi og en maksv�rdi.
Hvis en pixel i billedet har en intensitet h�jere end threshold v�rdien, tildeles denne pixel i resultatet til maksv�rdien.\cite{EmguCVLibDoc}

$$\text{dst}(x,y) = 
\begin{cases} 
max_{val} & \text{if src}(x,y) > threshold \\
0 & \text{if src}(x,y) \leq threshold
\end{cases}
$$

\subsubsection{Metoden AbsDiff}
Metoden tager et andet billede som input og beregner et billede
som er den absolutte differens mellem de to billeder s�ledes at.
\cite{EmguCVLibDoc}

$$ \text{dst}(x,y) = \vert \text{imgA}(x,y) - \text{imgB}(x,y) \vert $$

\subsection{Metoden ApproxPoly}
Denne metode generere en aproximation af konturen med f�rre punkter ved hj�lp af Ramer?Douglas?Peucker algoritmen.
\cite{opencvrefman}[s.~269]
Som input tager den en \emph{double} der bruges til at definere resultatets pr�sition.
\cite{EmguCVLibDoc}
