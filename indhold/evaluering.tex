Dette kapitel evaluerer de to valgte sensormodeller(\cref{mapping:sensormodel}) og ruteplanlægning(\cref{ruteplanleagning}).

\section{Formål}
Formålet med denne test er at se hvilken sensormodel der kommer frem til det mest præcise kort og hvilken indvirkning ruteplanlægningen har.
Dette er for at se om målsætningen er opfyldt beskrevet i \cref{problem:maalsaetning}.

\section{Test}\label{evaluering:test_beskrivelse}
Der bliver foretaget tre tests med hver sensormodel(\cref{mapping:sensormodel}).
I alle test benyttes ruteplanlægning beskrevet i \cref{ruteplanleagning}.
Robotten kører hen til et punkt og scanner to gange - dette foretages 75 gange.
Alle data bliver logget så det er muligt at genskabe et kort udfra det.
Opstillingen af testen er beskrevet i \cref{evaluering:opstilling}

\subsection{Opstilling}\label{evaluering:opstilling}
Det brugte testmiljø er beskrevet i \cref{testmiljo}.
Selve opstillingen til testen kan man se på \cref{evaluering:emptyGrid}, her kan man desuden se robottens startposition, som er den samme i alle tests.

\begin{figure}[h]
\includegraphics[width=\textwidth]{emptyGrid}
\label{evaluering:emptyGrid}
\caption{Forsøgsopstillingen inden hver test sættes i gang.}
\end{figure}
\subsection{Resultater}

\subsection{Opsummering}