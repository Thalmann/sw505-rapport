\section{System arkitektur}\label{arkitektur}
Til dette projekt er det besluttet at bygge systemet efter en lag-delt struktur.
Denne beslutning er taget for at for at overkomme kompleksiteten af systemet; altså, for at give hvert gruppemedlem bedre overblik over koden, men også for at skabe nogle generelle retningslinjer for placering at dets komponenter.

I de efterfølgende afsnit, vil arkitekturen (som ses på  \cref{arkitektur:klassediagram:1,arkitektur:klassediagram:2}) blive beskrevet.
Beskrivelsen vil fokusere på hvad de enkelte lag (namespaces) har til formål samt hvordan lagene interagerer med hinanden.
Som det også ses på figurene, er det kun de ydre associationer der er angivet for at simplificere beskrivelsen af systemet.
Som det første beskrives \textbf{CommonLib.dll} i \cref{arkitektur:commonlib}, da systemets lag alle holder referencer hertil.
Efterfølgende vil lagene blive beskrevet i en rækkefølge der følger kaldene internt i systemet.

\thilemann{Overvej om de to figurer bliver for utydelige - skal de sættes op anerledes?}
\begin{figure}
\centering
\includegraphics[width=1\textwidth]{./graphics/systemarkitektur_1}
\caption{Lagene i systemarkitekturen med deres ydre associationer. Associationerne til højre fører alle til \textbf{CommonLib.dll}, som ses på \cref{arkitektur:klassediagram:2}.}
\label{arkitektur:klassediagram:1}
\end{figure}

\begin{figure}
\centering
\includegraphics[width=1\textwidth]{./graphics/systemarkitektur_2}
\caption{Viser namespacet \textbf{CommonLib.dll} (uden interne associationer) som er en del af den lagdelte systemarkitektur. De fleste lag i arkitekturen har associationer til netop \textbf{CommonLib.dll}, der indeholder de elementer som benyttes af alle lag. De refererende namespaces kan ses på \cref{arkitektur:klassediagram:1}.}
\label{arkitektur:klassediagram:2}
\end{figure}

\subsection{CommonLib.dll}\label{arkitektur:commonlib}
Benyttes af alle lag i arkitekturen.
Dets formål er at samle de komponenter som i systemet kan benyttes af alle lag.
Derfor holder det også en reference til namespacet \textbf{NKH.MindSqualls} for at kunne tilgå den del af MindSqualls (\cref{mindsqualls}) som indeholder kommunikationen til NXT-enheden.

\subsubsection{CommonLib.DTOs}


\subsubsection{CommonLib.NXTPostMan}

\subsubsection{CommonLib.Interfaces}

\subsection{NKH.MindSqualls}\label{arkitektur:mindsqualls}


\subsection{SystemInterface}\label{arkitektur:systeminterface}
Det er fra \textbf{SystemInterface} namespacet at 

\subsubsection{SystemInterface.GUI.exe}

\subsubsection{SystemInterface.RobotInterface}

\subsubsection{SystemInterface.ConsoleUI.exe}
\thilemann{Hvad med denne??}

\subsection{Control.dll}\label{arkitektur:control}


\subsubsection{Control.DisplayControl}

\subsubsection{Control.LocationControl}

\subsubsection{Control.MappingControl}


\subsection{Services.dll}\label{arkitektur:services}


\subsubsection{Services.RobotServices}

\subsubsection{Services.RouteService}

\subsubsection{Services.KinectServices}

\subsubsection{Services.TrackingServices}


\subsection{Data.dll}\label{arkitektur:data}







