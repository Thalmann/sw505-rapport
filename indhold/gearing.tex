\section{Gearing}\label{robot:gearing}
I \cref{sensorer:motorer} fandt vi ud af at præcisionen på motorerne ikke var høj.
Der var en afvigelse på op til 4\dg, imod den maksimale afvigelse på 1\dg ~nævnt i kravene i \cref{robot:design}.
Dette faktum gav anledning til at udforske mulighederne for at geare motorene for at mindske denne præcision.
\thilemann{Vi mindsker vel usikkerhederne og \textit{øger} præcisionen?}

\subsection{Simpel Teori}\label{gearing:simpel_teori}
Gearing kan foregå på to måder; geare op eller geare ned.
Det tandhjul der er knyttet direkte til motoren kalder vi fører-tandhjulet og det tandhjul der er knyttet til fører-tandhjulet kalder vi for følger-tandhjulet (se evt. \cref{gearing:nedgearing}).

\subsubsection{Nedgearing}
Nedgearing foregår ved at et mindre tandhjul driver et større tandhjul.
Gear rationen (størrelsen af gearing) er styret af antallet af tænder på tandhjulene.
For eksempel vil et 24-tands fører-tandhjul drive et 40-tands følger-tandhjul med ratioen $1:1.667$, hvilket betyder at der for at give en enkelt følger omdrejning kræves $1.667$ fører omdrejninger. 
Det betyder at motoren der driver fører-tandhjulet skal rotere $\frac{40}{24} = 1 \frac{2}{3}$ omgange for at rotere følger-tandhjulet én omgang og at følger tandhjulet roterer $\frac{1}{1.667} = 0.6$ omgange pr. omgang af fører tandhjulet.


\begin{figure}[h]
\centering
\includegraphics[width=.5\textwidth]{gears/op_og_ned}
\caption{Eksempel på (ned-)gearing}
\label{gearing:nedgearing}
\end{figure}

\subsection{Aktuelle gearing}
Dette afsnit fokuserer på den teoretiske gearing for de ultrasoniske sensorer.
\Cref{robot:gearing-test} beskriver forskellige test af gearingerne der driver sensor rotationen for at undersøge om gearingen lever op til de teoretiske værdier.


\subsubsection{Ultrasonisk sensor}
Denne bruges til at bestemme afstanden til et objekt i en bestemt retning, hvorfor det vil være en klar fordel at geare motoren ned for at opnå større præcision af denne når retningen af sensoren skal bestemmes.
Rotationen vil naturligvis foregå langsommere end uden gearing, men da tid ikke er en faktor på nuværende tidspunkt er det ikke af nogen betydning for at løse problemet.

Gearingen for den ultrasoniske sensor består af i alt 4 tandhjul, som alle kan ses i \cref{gearing:tandhjul}.

\begin{figure}[h] % De anvendte tandhjul
\centering
\begin{subfigure}[b]{.19\textwidth}
\centering
\includegraphics[width=\textwidth]{gears/worm}
\caption{Snekke}
\label{gearing:snekke}
\end{subfigure}
%\begin{subfigure}[b]{.19\textwidth}
%\centering
%\includegraphics[width=\textwidth]{gears/16-tooth}
%\caption{16-tands}
%\label{gearing:16tand}
%\end{subfigure}
\begin{subfigure}[b]{.19\textwidth}
\centering
\includegraphics[width=\textwidth]{gears/24-tooth}
\caption{24-tands}
\label{gearing:24tand}
\end{subfigure}
\begin{subfigure}[b]{.19\textwidth}
\centering
\includegraphics[width=\textwidth]{gears/40-tooth}
\caption{40-tands}
\label{gearing:40tand}
\end{subfigure}
\caption{De anvendte tandhjul til sensor rotation}
\label{gearing:tandhjul}
\end{figure}

Den første kombination består af en snekke\cite{snekke} (se \cref{gearing:snekke}) som fører-tandhjul og 24-tands (se \cref{gearing:24tand}) som følger-tandhjul.
Snekken kræver en hel rotation for at flytte én tand på følger-tandhjul.
Dette giver en gear ratio på $1:24$, hvilket betyder at der kræves 24 hele motor-rotationer for at rotere 24-tands (følger) tandhjulet én omgang.

Den anden kombination består af en 24-tands (se \cref{gearing:24tand}) som fører-tandhjul og en 40-tands (se \cref{gearing:40tand}) som følger-tandhjul.
Dette giver en gear ratio på $1:1.667$.

Den samlede gear ratio for sensoren bliver derfor $1:40$, som beskriver at der for hver sensor omdrejning kræves 40 motor omdrejninger, hvilket er det samme som:
$$\frac{24}{1} \cdot \frac{40}{24} = 40$$

\subsection{Test af gearing}\label{robot:gearing-test}
Nu hvor den teoretiske gearing er fastlagt, vil der her blive udført en række forsøg for at afgøre præcisionen af sensor motoren med gearing; og evt. at fastlægge en faktisk gearing, hvis denne afviger fra den teoretiske.

\subsubsection{Sensor}
Denne test er meget vigtigere end hjul-testen, da det er vigtigt at vide den nøjagtige rotation af sensoren, og dermed dens retning, for at få så præcise målinger som muligt.

\paragraph{Første test} var en grovtest, hvor der blev kørt i den samme retning i et antal sensor-omdrejninger.
Igen blev der udført handling med sensor-omdrejninger som input, ud fra en beregnet motor-omdrejning, baseret på den teoretiske gearing.
Denne test gav et godt resultat, uden afvigelse.
\thilemann{en forsker udførte engang en meget hemmelighedsfuld test af en gearing helt ude i skoven, uden nogen former for resultater...}

\paragraph{Anden test} blev udført ved at køre færre omdrejninger, skiftevis med og mod uret.
Dette gav en upræcision, da der er slør i den øverste gearing, mellem 56-tands og 24-tands tandhjulene.
Dette slør giver mellem $\frac{1}{4}$ og $\frac{3}{4}$ tands unøjagtighed, da der ved skift af rotations-retning bruges op til $\frac{3}{4}$ tandhjul-omdrejning for 24-tands tandhjulet at få fat i 56-tands tandhjulets tænder igen.
\thilemann{Denne test skal genudføres og der skal skrives lidt om den nye løsning - man kunne evt. skrive at det er ændret pga. sløret (afsnit om sløret kan findes på git...)}