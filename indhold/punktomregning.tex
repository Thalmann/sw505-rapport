\section{Omregning fra punkt i billede til reelt punkt}\label{lokalisering:punktomregning}
Det billede der fås ind fra kameraet, består af en mængde pixels.
Robottens centrum, og derved lokation, vil være i præcist én pixel.
For at robotten skal kunne bruge denne til at navigere i den virkelige verden, er det nødvendigt at omregne pixel(-koordinat) til reelt koordinat.

\subsection{Ensvinklede trekanter}
Den overordnede idé er at bruge reglen om ensvinklede trekanter, da det passer på situationen som kan ses i \cref{fig:kameratrekant}.
Her er $v$ den synsvinkel som kameraet har i den givne dimension; for Kinect'en er det $57^\circ$ i bredden og $43^\circ$ i højden.
$v$ er blot med for at vise at der er tale om ensvinklede trekanter.
$x_1$ er størrelsen af den givne dimension for det billede der dannes af Kinect'en.
$x_2$ svarer til $x_1$, blot den reelle størrelse af det billede der gives.
\mikael{Der er underlige sætninger her der skal skrives om}

\begin{figure}[h]
\centering
\begin{tikzpicture}[extline/.style={shorten >=-#1,shorten <=-#1},
  extline/.default=1cm]

\coordinate (A) at (0,0);
\coordinate (B) at (2,6);
\coordinate (C) at (4,0);
\coordinate (A2) at (1,3);
\coordinate (C2) at (3,3);

\draw [dashed] (A) -- (B);
\draw [dashed] (B) -- (C);
\draw (A) -- (C) node [midway,below] {$x_2$};
\draw (A2) -- (C2) node [midway,below] {$x_1$};
\draw ($(B) + (-110:0.55)$) arc (-110:-70:0.55) node [right] {$v$};
\end{tikzpicture}
\caption{Situationen med de kendte variable}
\label{fig:kameratrekant}
\end{figure}

\begin{description}
\item[Sætning:]{Hvis alle vinkler i to trekanter er parvis lige store, er siderne parvis proportionale.}
\end{description}
Som det kan ses i \cref{fig:udregningstrekant} er der tale om netop sådan en trekant for vores situation.
Trekanten er halveret for at midten får koordinatet $(0,0)$, derved går en dimension fra $-\frac{1}{2}x$ til $+\frac{1}{2}x$.

\subsection{Beregning}
For at finde proportionalitetsfaktoren deles en vilkårlig side af den store trekant med den tilsvarende side fra den lille trekant.
Her vil proportionalitetsfaktoren derved blive $f = \frac{\frac{1}{2}x_2}{\frac{1}{2}x_1}$.

For at omregne et pixel-koordinat $p = (p_1, p_2)$ om til et reelt koordinat, skal der blot påregnes proportionalitetsfaktoren.
Så derved får vi vores reelle koordinat $k = f \cdot p$.
For vores sitation er der 2 dimensioner; længde og bredde: \begin{equation*}
(k_1,k_2) = (f \cdot p_1, f \cdot p_2)
\end{equation*}


\begin{figure}[h]
\centering
\begin{tikzpicture}[extline/.style={shorten >=-#1,shorten <=-#1},
  extline/.default=1cm]

\coordinate (A) at (0,0);
\coordinate (B) at (0,6);
\coordinate (C) at (2,0);
\coordinate (A2) at (0,3);
\coordinate (C2) at (1,3);
\coordinate (mAC) at (1.25,0);
\coordinate (oldA) at (-2,0);

\draw [dashed,color=gray] (oldA) -- (B);
\draw [dashed,color=gray] (oldA) -- (A);
\draw [thick,dotted] (mAC) -- (B);
\draw (A) -- (B);
\draw (B) -- (C);
\draw (A) -- (C) node [below,midway,xshift=-1cm] {$0$} node [below,midway,xshift=1.2cm] {$\frac{1}{2}x_2$};
\draw (A2) -- (C2) node [midway,xshift=1cm] {$\frac{1}{2}x_1$} node [midway,xshift=-0.7cm] {$0$};
\draw ($(B) + (-90:0.55)$) arc (-90:-70:0.55) node [right] {$\frac{1}{2}v$};
\filldraw (0.63,3) circle (0.05cm) node [left,yshift=-0.3cm] {$p$};
\filldraw (1.25,0) circle (0.05cm) node [left,yshift=-0.3cm] {$k$};
\end{tikzpicture}
\caption{Selve omregningen}
\label{fig:udregningstrekant}
\end{figure}