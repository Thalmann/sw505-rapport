\section{Notation for probabilistisk robotstyring}
Da det er ønskværdigt at kunne give en præcis beskrivelse af robottens positur, dvs. dens position og retning, vil dette afsnit kort introducere den nødvendige notation, som foreslået af Sebsatian Thrun \citep[s. 16-21]{probabilisticRobotics}.

\begin{itemize}
\item \textbf{Tilstand} betegner den tilstand miljøet er i; altså, robottens positur, omkringliggende objekter som vægge, bygninger osv. 
Tilstand kan være \textit{dynamisk} (tilstanden kan ændre sig -- fx. position for en person) og \textit{statisk} (tilstanden ændrer sig ikke -- fx. position for en bygning).
Tilstand beskrives af variablen $x$, som også indeholder information omkring robotten selv, f. eks. dens \textit{positur}, \textit{hastighed} og dens \textit{sensorer}.

\item \textbf{Tilstand i tiden} $\mathbf{t}$ betegnes af variablen $x_t$ og beskriver den seneste \textit{kendte} tilstand. 
Den forrige seneste måling angives med $x_{t-1}$ og målingen efter den seneste som $x_{t+1}$.

\item \textbf{Målingsdata} indeholder information om robottens omgivelser til et bestemt tidspunkt. 
$z_t$ er således målingsdata til tiden \textit{t}. 
Notationen

$z_{t_1:t_k} = z_{t_1}, z_{t_2}, z_{t_3}, \dots , z_{t_k}$

betegner alle målinger fra tiden \textit{$ t_1 $} til tiden \textit{$ t_k $}
\item \textbf{Kontrol data} indeholder information om ændring af robottens tilstand. 
Kontrol data kan for eksempel være robottens hastighed, eller en aflæsning af en motors odometer, der fortæller hvor mange omdrejninger hjulet har foretaget.
$u_t$ betegner ændringen af robottens tilstand i intervallet fra \textit{t-1} til \textit{t}.
Igen betegner notation

$u_{t_1:t_k} = u_{t_1}, u_{t_2}, u_{t_3}, \dots , u_{t_k}$

mængden af kontrol data fra \textit{$ t_1 $} til \textit{$ t_k $}.
\end{itemize}