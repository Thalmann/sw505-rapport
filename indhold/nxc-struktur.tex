% !TeX spellcheck = da_DK
\section{NXT Software}
Koden, der kører på robotten, er bygget op omkring et modul, der hedder MotorControl. \cite{MotorControl}
MotorControl er et program, der kører på NXTen og sørger for at motorerne kører præcis den afstand, de bliver bedt om. 
Uden MotorControl er motorerne meget upræcise, hvilket gør det svært at navigere robotten til et bestemt sted.

Da der kun kan køre ét program ad gangen på NXTen, indeholder vores program en modificeret version af MotorControl, hvor alle navigationsfunktioner udnytter MotorControl.

Koden på robotten er opdelt i filer efter hvilket ansvar de har.
I det følgende vil hver fil blive beskrevet.

\subsection{MotorControl}
For at MotorControl kan fungere, er der inkluderet følgende filer, der ikke er ændrede i forhold til kilden.

\begin{itemize}
\item \textbf{SpeedFromPosLookup.nxc}
\item \textbf{MotorFunctions.nxc}
\item \textbf{ControllerCore.nxc}
\item \textbf{Controller.nxc}
\end{itemize}

Filen \textbf{MCTasks.nxc} indeholder de tasks, der er i den originale MotorControl, men med ændringer, som gør dem brugbare i vores program.
Den indeholder desuden en \lstinline[style=c]!Run! metode, som sætter MotorControl igang. 
Denne Run funktion var i det originale MotorControl en del af et switch statement i \lstinline[style=c]!Main! funktionen, og er derfor trukket ud, så det er en funktion, der kan kaldes.

\subsection{NXC koden}

Programmet samles i filen \textbf{MainTask.nxc}, som indeholder den løkke, der sættes i gang og efterfølgende køres kontinuerligt ved program-start. 
I denne løkke venter robotten på komandoer fra computeren og reagerer derefter på dem.

Alt hvad robotten skal gøre håndteres i en af disse tre filer:

\begin{description}
\item[Navigation.nxc]{styrer robottens generelle navigation, fx \lstinline[style=c]!RunForward!, der beder robotten køre fremad, \lstinline[style=c]!TurnAngle!, der drejer robotten og \lstinline[style=c]!GoToPoint!, der beder robotten køre hen til et bestemt punkt.}
\item[Sensor.nxc]{bliver brugt hver gang der skal foretages en sensormåling.}
\item[Communication.nxc]{bliver brugt til at kommunikere med computeren med funktionerne \lstinline[style=c]!IAmThere!, når robotten har nået sin destination og funktionen \lstinline[style=c]!WhereAmI!, når robotten forespørger sin position.}
\end{description}
