\subsection{NXC}\label{nxc}
Til at programmere robotten (NXTen) har vi valgt at benytte NXC; et programmeringssprog designet til NXT robotten. 
Brugen af NXC til at styre robotten vil blive beskrevet her.\cite{NXC}
Desuden bruges Enhanced Firmware, for at få adgang til trigonometriske funktioner, som er nødvendige for projektet.\cite{enhanced_firmware}

\subsubsection{NXC sproget}
NXC har en syntax der minder meget om C syntax. 
Ud over de traditionelle programmeringskonstruktioner indeholder NXC en masse standardfunktioner til at interagere med robottens sensorer og motorer.

\subsubsection{Sensorer og motorer}
NXC sproget indeholder funktioner til at styre sensorer og motorer tilsluttet NXTen.
Et eksempel på aktivering af en sensor og en motor kan ses i \cref{nxc:sensors}.
\lstinline[style=c]!SetSensor! bruges først til at fortælle robotten at indgang 1 er en touch sensor. 
Sensorens værdi aflæses i linje 5 fra en variabel tilknyttet porten.
Motorne i port A og C instrueres til at køre med 75 \% hastighed i linje 4 med \lstinline[style=c]!OnFwd!.  
Der findes tilsvarende en \lstinline[style=c]!OnRev! til at køre baglæns.
Motoren kører indtil der trykkes på touch sensoren, hvorefter \lstinline[style=c]!Off! kaldes for at standse motorerne.

\begin{lstlisting}[style=c,label=nxc:sensors, caption={Brug af motorer og sensorer.}]
task main()
{
 SetSensor(IN_1,SENSOR_TOUCH);
 OnFwd(OUT_AC, 75);
 until (SENSOR_1 == 1);
 Off(OUT_AC);
}
\end{lstlisting}

\subsubsection{Tasks}
\mikael{Så vidt jeg ved bruger vi ikke dette, så mon ikke bare det bør undlades?}
\bruno{enig med mikael}
Til at udnytte at NXT robotten understøtter multithreading, findes der i NXC et \lstinline[style=c]!task! keyword.
Et NXC program består af maksimalt 255 tasks, der hver har et unikt navn. 
Der skal altid findes en \lstinline[style=c]!Main! task, da denne køres først.
Main task'en skal terminere før andre tasks kan køres, og tasks skal enten køres eksplicit eller planlægges (schedules).

Tasks kan schedules på en af følgende måder:

\begin{description}
\item{\lstinline[style=c]!Precedes(task1,task2,...,taskN!)}

Erklærer at denne task kommer før de angivne tasks. 
De angivne tasks vil blive sat til at køre samtidigt, når den nuværende task afsluttes.
\item{\lstinline[style=c]!Follows(task1,task2,...,taskN!)}

Erklærer at denne task kommer efter de angivne tasks. 
De angivne tasks skal afsluttes før den nuværende task får lov til at køre.
\end{description}

Tasks kan startes og stoppes med de følgende kommandoer:

\begin{description}
\item{\lstinline[style=c]!StartTask(task t)!}

Starter den angivne task

\item{\lstinline[style=c]!StopTask(task t)!}

Stopper den angivne task
\end{description}

Der er i sproget indbygget en \lstinline[style=c]!mutex! datatype til brug ved samtidig programmering, samt en \lstinline[style=c]!Acquire! og en \lstinline[style=c]!Release! funktion.
Et eksempel på brug af mutex kan ses i \cref{nxc:mutex}

\begin{lstlisting}[style=c,label=nxc:mutex, caption={Eksempel på brug af mutex.}]
mutex moveMutex;

task move_square()
{
  while (true)
    {
      Acquire(moveMutex);
      OnFwd(OUT_AC, 75); Wait(1000);
      OnRev(OUT_C, 75); Wait(500);
      Release(moveMutex);
    }
}

task check_sensors()
{
  while (true)
  {
    if (SENSOR_1 == 1)
    {
      Acquire(moveMutex);
      OnRev(OUT_AC, 75); Wait(500);
      OnFwd(OUT_A, 75); Wait(500);
      Release(moveMutex);
    }
  }
}

task main()
{
  Precedes(move_square, check_sensors);
  SetSensorTouch(IN_1);
}
\end{lstlisting}

\subsubsection{Kommunikation med PC}
Da der skal bruges en ekstern enhed (en PC) til visse dele af systemet, er det nødvendigt at kunne kommunikere med en PC fra NXTen.
Til kommunikation via Bluetooth findes der på NXT et antal mailboxe.
Disse anvendes til at sende og modtage beskeder.
Et eksempel på at modtage en besked kan ses i \cref{nxc:receive}.
\lstinline[style=c]|ReceiveRemoteString| bruges til at kigge om der er en ny besked i \lstinline[style=c]!INBOX!.
Denne bliver fjernet fra inboxen, hvis det andet parameter er \lstinline[style=c]|true|, og beskeden bliver læst over i strengen \lstinline[style=c]|in|.

\begin{lstlisting}[style=c,label=nxc:receive,caption={Et eksempel på at modtage beskeder over Bluetooth.}]
#define INBOX 5
#define OUTBOX 6

while(true){
  ReceiveRemoteString(INBOX,true,in)
  
  if(StrLen(in) > 0)
  {
    tmp = SubStr(in,0,1);
    packet = StrToNum(tmp);
    PlayToneEx(100* packet, 200, 3,false);
  }
  in = "";
}
\end{lstlisting}

I \cref{nxc:receive-mindsqualls} er den tilhørende kode i \mindsqualls, der afsender en besked.
Først gemmes beskeden i en streng og derefter afsendes beskeden via \lstinline[style=csharp]!NxtCommunicationProtocol! klassens \lstinline[style=csharp]!MessageWrite! funktion.

\begin{lstlisting}[style=c,breaklines=true, label=nxc:receive-mindsqualls,caption={\mindsqualls kode der afsender en besked.}]
private const NxtMailbox2 PC_INBOX = NxtMailbox2.Box6;
private const NxtMailbox PC_OUTBOX = NxtMailbox.Box5;

public static void NxtTone(NxtCommunicationProtocol commLink, int tone)
{
  string messageData = string.Format("{0}", (byte)tone);
  commLink.MessageWrite(PC_OUTBOX, messageData);
}
\end{lstlisting}

Afsending af beskeder virker tilsvarende ved brug af kommandoen \lstinline[style=c]|SendMessage|.
Et eksempel på at afsende en besked kan ses i \cref{nxc:send}.
\lstinline[style=c]|SendMessage| benyttes til at sende beskeden \lstinline[style=c]|msg| til \lstinline[style=c]!OUTBOX!.

\begin{lstlisting}[style=c,label=nxc:send,caption={Eksempel på afsending af besked.}]
while(true){
  string msg = NumToStr(SENSOR_1);
  SendMessage(OUTBOX, msg);
}
\end{lstlisting}

I \cref{nxc:send-mindsqualls} ses den tilhørende \mindsqualls kode, der læser en besked sendt fra NXTen.

\begin{lstlisting}[style=c,breaklines=true,label=nxc:send-mindsqualls,caption={\mindsqualls kode der læser en besked sendt fra NXT'en.}]
public static string NxtSensorReading(NxtCommunicationProtocol commLink)
{
  return commLink.MessageRead(PC_INBOX, NxtMailbox.Box0, true);
}
\end{lstlisting}
