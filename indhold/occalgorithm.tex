% !TeX spellcheck = da_DK
\section{Occupancy Grid algoritmen}
Dette afsnit vil gennemgå algoritmen, der skal opdatere et occupancy grid.
Algoritmen er baseret på occupancy algoritmen fra \cite[p. 286]{probabilisticRobotics}.
Formålet med \textit{occupancy grid} algoritmen er, at beregne	et nyt kort ud fra et kort og en samling data:

\begin{equation}
p(m \mid z_{1:t}, x_{1:t})
\end{equation}

hvor $m$ er et kort, $ z_{1:t} $ er mængden af målinger op til tiden $t$ og $ x_{1:t} $ er robottons rute i form af en følge af \textit{poses}.

Kortet er inddelt i et finit antal celler.
Hver celle kan findes ved $ m_i $ hvor $i$ er cellens indeks. 
Hele kortet kan derfor betegnes

\begin{equation}
m = \sum_{i}^{} m_i 
\end{equation}

Hver celle er tilknyttet en binær værdi der betegner hvorvidt cellen er \textit{occupied} eller \textit{free}.
Værdien 1 betegner \textit{occupied}, mens 0 betegner \textit{free}.

For at begrænse mængden af celler, der skal beregnes ved en opdatering af et occupancy grid, nedbrydes problemet til en mængde delproblemer af formen

\begin{equation}
p(m_i \mid z_{1:t}, x_{1:t})
\end{equation} 

Da der nu er et problem af denne type for hver celle, kan man nøjes med kun at opdatere de celler, der er ny information om. 
Det er således kun de celler, som er indenfor robottens synsfelt, dvs. de celler som ligger i samme række eller kolonne som robotten i vores occupancy grid, der skal opdateres.


Som i det originale binære filter (se evt. \cref{bayes_binaerfiltre}) benyttes log odds repræsentationen til at udtrykke om en celle er \textit{occupied}.
$l_{t,i}$ repræsenteree sandsynligheden til tiden $t$ for at cellen $m_i$ er \textit{occupied} på logodds form og $l_r$ er værdien fra vores sensor model. \\

$l_{t,i}$ opdateres på følgende måde.
\begin{equation}
	l_{t,i} = \begin{cases}
		l_{t-1,i} + l_r - l_0 &\text{hvis } x_i = x_r \vee y_i = y_r \\
		l_{t-1,i} &\text{ellers}
	\end{cases}
\end{equation}
Hvor $x_i, y_i$ er grid koordinatet for cellen $m_i$ mens $x_r, y_r$ er grid koordinatet for robotten. Hvilket giver.

\begin{equation}
l_{t,i} = log{ \frac{p(m_i \mid z_{1:t}, x_{1:t})}{1 - p(m_i \mid z_{1:t}, x_{1:t})}} 
\end{equation} 

Sandsynligheden genfindes ud fra log odds repræsentationen ved

\begin{equation}
	p(m_i \mid z_{1:t}, x_{1:t}) = 1 - \frac{1}{1 + e^{l_{t,i}}}
\end{equation} 

\Cref{occupancygrid:alg} viser hvordan cellerne i vores occupancy grid opdateres udfra en sensormåling. 
Til opdateringen bruges \textbf{inverse\_sensor\_model}, \cref{alg:inversesensormodel}.
Værdien $ l_0 $ er den a priori værdi for, om cellen er optaget repræsenteret som log odds udfra \cref{eqn:l0}.

\begin{equation}
  l_0 =  log \frac{p(m_i = 1)}{p(m_i = 0)} = log \frac{p(m_i)}{1- p(m_i)}
\end{equation} 

\begin{algorithm}[H]
\LinesNumbered
OccupancyGridMapping(\{$l_{t-1,i}$\}, $x_t$, $z_t$):

\ForAll{cells $ m_i $}
{
\eIf{$ x_i = x_r$ or $y_i = y_r$}
%then
{ $ l_{t,i} = l_{t-1,i} $ + \textbf{inverse\_sensor\_model} $( m_i, x_t, z_t ) - l_0$\\ }
%else
{ $ l_{t,i} = l_{t-1,i}  $\\ }
}
\Return {$ \{l_{t,i}\} $}
\caption{Occupancy grid opdateringsalgoritmen.}
\label{occupancygrid:alg}
\end{algorithm}




