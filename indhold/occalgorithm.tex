% !TeX spellcheck = da_DK
Formålet med \textit{occupancy grid} algoritmen er at beregne	et nyt kort ud fra et kort og en samling data:

\[ p(m \mid z_{1:t}, x_{1:t}) \]

hvor m er et kort, $ z_{1:t} $ er mængden af målinger op til tiden t og $ x_{1:t} $ er robottons rute i form af en følge af \textit{pose}.

Kortet er inddelt i et finit antal celler.
Hver celle kan findes ved $ m_i $ hvor i er cellens indeks. 
Hele kortet kan derfor betegnes

\[  m = \sum_{i}^{} m_i \]

Hver celle er tilknyttet en binær værdi der betegner hvorvidt cellen er \textit{occupied} eller \textit{free}.
Værdien 1 betegner \textit{occupied} mens 0 betegner \textit{free}.

For at begrænse mængden af celler der skal beregnes ved en opdatering af et occupancy grid nedbrydes problemet til en mængde delproblemer af formen

\[ p(m_i \mid z_{1:t}, x_{1:t}) \]

