\section{Lokalisering}
N�r robotten skal kortl�gge sine omgivelser er det n�dvendigt at vide hvor robotten befinder sig.
Til at finde robottens position har vi overvejet flere forskellige metoder.
\begin{itemize}
\item Kinect
\item Dead reckoning
\item Kamera ansigtsgenkendelse
\end{itemize}

\subsection{Kinect }
Microsofts Kinect benytter et rgb kamera samt en dybdesensor til at fortolke bev�gelse som input i applikationer.
Kinect er oprindeligt lavet til XBox 360, men findes ogs� i en version til PC, med medf�lgende SDK.
En stor fordel ved at bruge Kinect er at den indeholder en dybdesensor der kan bruges til at lokalisere objekter i et rum.

\subsection{Ansigtsgenkendelse}
En anden mulighed er at benytte et kamera og f�lge robotten med ansigtsgenkendelsessoftware.


\subsection{Dead reckoning}
Dead Reckoning er en teknik til at finde sin position efter ens bev�gelser uden eksterne sensorer.
Robotten ved hvor den starter, og beregner sin nye position efter hvilken kommando den har udf�rt.\cite{deadrec}

Denne teknik kan risikere at v�re meget upr�cis, hvis robottens motorer og sensorer mangler pr�cision.

\subsection{Valg af lokaliseringsmetode}
Der blev foretaget en indledende test af de tre foresl�ede lokaliseringsmetoder for at v�lge hvilken der passede bedst til dette projekt.

Dead reckoning muligheden blev hurtigt 